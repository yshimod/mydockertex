\documentclass[uplatex,11pt]{jsarticle}
\usepackage[sourcehan]{pxchfon}
\usepackage[top=25truemm,bottom=10truemm,left=15truemm,right=15truemm,includefoot]{geometry}
\setlength{\headsep}{10truemm}
\setlength{\textheight}{250truemm}

\usepackage{fancyhdr}
\usepackage{lastpage}

\usepackage[dvipdfmx]{graphicx}
\usepackage{amsmath}
\usepackage{amssymb}
\usepackage{ascmac}
\usepackage{amsthm}
\usepackage{mathrsfs}
\usepackage{ulem}
\usepackage{fancybox}
\usepackage{cancel}
\usepackage{plext}
\usepackage{array}
\usepackage{url}
\usepackage[dvipdfmx,svgnames]{xcolor}
\usepackage{tikz}
\usepackage{accents}
\usepackage{tcolorbox}


\newcommand{\OBITOP}{
	\colorbox{black}
		{\vbox 
			{
				\hsize163mm
				\textsf{ \textcolor{white}{
					平成29年度 春・夏学期 「経済・経営数学」
					\hspace*{\stretch{1}}
					平成29年6月7日提出
					}
				}
				\vfil \vspace{-.5mm}
			}
		}
}

\newcommand{\OBITITLE}{
	\vspace{-2mm}
	{\vbox 
		{
			\hspace{\stretch{1}}
			{\LARGE \textsf{ 
				レポート課題 1
				}
			}
			\hspace{\stretch{1}} \vfil
		}
	}
}

\newcommand{\OBINAME}{
	\vspace{-2mm}
	\colorbox{black}
		{\vbox 
			{
				\hsize163mm
				\vspace{-.5mm}
				\hspace*{\stretch{1}}
				\textsf{ \textcolor{white}{
					経済学専攻\ MC1\ 
					下平 勇太\ 
					{\scriptsize 学籍番号}:23A17018
					}
				}
				\vfil \vspace{-.7mm}
			}
		}
}

\newcommand{\OBIZERO}{
	\hspace{-2.75mm}
	\vbox{
		\hbox{\vspace{-40mm}}
		\hbox{\OBITOP\vspace{17mm}}
	}
}

\newcommand{\PAGEBANGO}{
\thepage{}/{}\pageref{LastPage}
}

\pagestyle{fancy}
\renewcommand{\headrulewidth}{0.0pt}
\renewcommand{\footrulewidth}{0.4pt}
\lhead{}
\chead{\vspace{5mm}\OBITITLE\OBINAME}
\rhead{}
\lfoot{}
\cfoot{\PAGEBANGO}
\rfoot{}


\newtcolorbox{irohako}[2][]{
	colframe=#1,
	colback=#1!5!white,
	colbacktitle=#1!30!white,
	coltitle=black,
	fonttitle=\sffamily,
	title=#2
}

\def\vekutoru#1{\mbox{\boldmath $#1$}}

\newcommand{\ubar}[1]{\underaccent{\bar}{#1}}

\newcommand{\bibun}{\mathrm{d}}

\newcommand{\maxprob}{\mathop\mathrm{maximise}\limits}
%\newcommand{\minprob}{\mathop\mathrm{min.}\limits}

\newcommand{\argumentmax}{\mathop\mathrm{arg\,max}\limits}

\def\shatai#1{\makebox[2.25zw][l]{\vphantom{#1}\rotatebox{-48.8}{\scalebox{0.875}[1.143]{\rotatebox{41.2}{\smash{\rlap{#1}}}}}}}

\def\maru#1{\textcircled{\scriptsize #1}}
\def\shoumon#1{\vspace{1em}\noindent\ovalbox{\textsf{ #1 }}}

\makeatletter
\renewcommand{\theequation}{\textit{\arabic{section}-\arabic{equation}}}
\@addtoreset{equation}{section}
\makeatother

\def\thesection{問題 1.\arabic{section}}

%%%%%%%%%%%%%%%%%%%%%%%%%%%%%%%%%%%%%%%%%%%%%%%%%%%%%%
\begin{document}
\noindent
\OBIZERO
\quad

%%%%%%%%%%%%%%%%%%%%%%%%%%%%%%%%%%%%%%%%%%%%%%%%%%%%%%
\section{}

\begin{shadebox}
	\begin{center}
		\texttt{01.pdf}
	\end{center}
\end{shadebox}
%\vspace{5mm}

\shoumon{(1) - 背理法}

$\mathscr{P}_1 \equiv \neg\alpha$;\quad
$\mathscr{P}_2 \equiv (\neg\alpha \Longrightarrow \alpha)$;\quad
$\mathscr{P}_3 \equiv (\mathscr{P}_2 \Longrightarrow \alpha)$.

%\begin{table}[h]
\begin{center}
\begin{tabular}{c|ccc} \hline\hline
		$\alpha$ & $\mathscr{P}_1$ & $\mathscr{P}_2$ & $\mathscr{P}_3$ \\ \hline
		T & F & T & \textbf{T} \\ 
		F & T & F & \textbf{T} \\ \hline\hline
\end{tabular} 
\end{center}
%\end{table}

$\alpha$の真偽がいずれの場合も,$(\neg\alpha \Longrightarrow \alpha) \Longrightarrow \alpha\ (\text{i.e.,}\ \mathscr{P}_3)$は真であり,この命題は恒真命題である.


\shoumon{(1) - 純粋仮言三段論法}

$\mathscr{P}_1 \equiv (\alpha \Longrightarrow \beta)$;\quad
$\mathscr{P}_2 \equiv (\beta \Longrightarrow \gamma)$;\quad
$\mathscr{P}_3 \equiv \mathscr{P}_1 \wedge \mathscr{P}_2$;\quad
$\mathscr{P}_4 \equiv (\alpha \Longrightarrow \gamma)$;\quad
$\mathscr{P}_5 \equiv (\mathscr{P}_3 \Longrightarrow \mathscr{P}_4)$.

\begin{center}
	\begin{tabular}{ccc|ccccc} \hline\hline
		$\alpha$ & $\beta$ & $\gamma$ & $\mathscr{P}_1$ & $\mathscr{P}_2$ & $\mathscr{P}_3$ & $\mathscr{P}_4$ & $\mathscr{P}_5$ \\ \hline
		T & T & T & T & T & T & T & \textbf{T} \\
		T & T & F & T & F & F & F & \textbf{T} \\
		T & F & T & F & T & F & T & \textbf{T} \\
		T & F & F & F & T & F & F & \textbf{T} \\
		F & T & T & T & T & T & T & \textbf{T} \\
		F & T & F & T & F & F & T & \textbf{T} \\
		F & F & T & T & T & T & T & \textbf{T} \\
		F & F & F & T & T & T & T & \textbf{T} \\ \hline\hline
	\end{tabular} 
\end{center}

$\alpha$,$\beta$,$\gamma$それぞれの真偽がいずれの場合であっても,$[(\alpha \Longrightarrow \beta) \wedge (\beta \Longrightarrow \gamma)] \Longrightarrow (\alpha \Longrightarrow \gamma)\ (\text{i.e.,}\ \mathscr{P}_5)$は真であり,この命題は恒真命題である.

\newpage
\shoumon{(1) - 構成的仮言三段論法}

$\mathscr{P}_1 \equiv (\alpha \Longrightarrow \beta)$;\quad
$\mathscr{P}_2 \equiv \alpha \wedge \mathscr{P}_1$;\quad
$\mathscr{P}_3 \equiv (\mathscr{P}_2 \Longrightarrow \beta)$.

\begin{center}
	\begin{tabular}{cc|ccc} \hline\hline
		$\alpha$ & $\beta$ & $\mathscr{P}_1$ & $\mathscr{P}_2$ & $\mathscr{P}_3$ \\ \hline
		T & T & T & T & \textbf{T} \\
		T & F & F & F & \textbf{T} \\
		F & T & T & F & \textbf{T} \\
		F & F & T & F & \textbf{T} \\ \hline\hline
	\end{tabular} 
\end{center}

$\alpha$,$\beta$,$\gamma$それぞれの真偽がいずれの場合であっても,$[\alpha \wedge (\alpha \Longrightarrow \beta)] \Longrightarrow \beta \ (\text{i.e.,}\ \mathscr{P}_3)$は真であり,この命題は恒真命題である.


\shoumon{(1) - 構成的仮言三段論法・否定形}

$\mathscr{P}_1 \equiv \neg\beta$;\quad
$\mathscr{P}_2 \equiv (\alpha \Longrightarrow \beta)$;\quad
$\mathscr{P}_3 \equiv \mathscr{P}_1 \wedge \mathscr{P}_2$;\quad
$\mathscr{P}_4 \equiv \neg\alpha$;\quad
$\mathscr{P}_5 \equiv (\mathscr{P}_3 \Longrightarrow \mathscr{P}_4)$.

\begin{center}
	\begin{tabular}{cc|ccccc} \hline\hline
		$\alpha$ & $\beta$ & $\mathscr{P}_1$ & $\mathscr{P}_2$ & $\mathscr{P}_3$ & $\mathscr{P}_4$ & $\mathscr{P}_5$ \\ \hline
		T & T & F & T & F & F & \textbf{T} \\
		T & F & T & F & F & F & \textbf{T} \\
		F & T & F & T & F & T & \textbf{T} \\
		F & F & T & T & T & T & \textbf{T} \\ \hline\hline
	\end{tabular} 
\end{center}

$\alpha$,$\beta$,$\gamma$それぞれの真偽がいずれの場合であっても,$[\neg\beta \wedge (\alpha \Longrightarrow \beta)] \Longrightarrow \neg\alpha \ (\text{i.e.,}\ \mathscr{P}_5)$は真であり,この命題は恒真命題である.


\shoumon{(2) - 分配律1}

$\mathscr{P}_1 \equiv \alpha \vee \beta$;\quad
$\mathscr{P}_2 \equiv \mathscr{P}_1 \wedge \gamma$;\quad
$\mathscr{P}_3 \equiv \alpha \wedge \gamma$;\quad
$\mathscr{P}_4 \equiv \beta \wedge \gamma$;\quad
$\mathscr{P}_5 \equiv \mathscr{P}_3 \vee \mathscr{P}_4$.

\begin{center}
	\begin{tabular}{ccc|ccccc} \hline\hline
		$\alpha$ & $\beta$ & $\gamma$ & $\mathscr{P}_1$ & $\mathscr{P}_2$ & $\mathscr{P}_3$ & $\mathscr{P}_4$ & $\mathscr{P}_5$ \\ \hline
		T & T & T & T & \textbf{T} & T & T & \textbf{T} \\
		T & T & F & T & \textbf{F} & F & F & \textbf{F} \\
		T & F & T & T & \textbf{T} & T & F & \textbf{T} \\
		T & F & F & F & \textbf{F} & F & F & \textbf{F} \\
		F & T & T & T & \textbf{T} & F & T & \textbf{T} \\
		F & T & F & T & \textbf{F} & F & F & \textbf{F} \\
		F & F & T & F & \textbf{F} & F & F & \textbf{F} \\
		F & F & F & F & \textbf{F} & F & F & \textbf{F} \\ \hline\hline
	\end{tabular} 
\end{center}

$\alpha$,$\beta$,$\gamma$それぞれの真偽がいずれの場合であっても,$(\alpha \vee \beta) \wedge \gamma\ (\text{i.e.,}\ \mathscr{P}_2)$と$(\alpha \wedge \gamma) \vee (\beta \wedge \gamma)\ (\text{i.e.,}\ \mathscr{P}_5)$は真偽が一致するため,この2つの命題は等価である.

\newpage
\shoumon{(2) - 分配律2}

$\mathscr{P}_1 \equiv \alpha \wedge \beta$;\quad
$\mathscr{P}_2 \equiv \mathscr{P}_1 \vee \gamma$;\quad
$\mathscr{P}_3 \equiv \alpha \vee \gamma$;\quad
$\mathscr{P}_4 \equiv \beta \vee \gamma$;\quad
$\mathscr{P}_5 \equiv \mathscr{P}_3 \wedge \mathscr{P}_4$.

\begin{center}
	\begin{tabular}{ccc|ccccc} \hline\hline
		$\alpha$ & $\beta$ & $\gamma$ & $\mathscr{P}_1$ & $\mathscr{P}_2$ & $\mathscr{P}_3$ & $\mathscr{P}_4$ & $\mathscr{P}_5$ \\ \hline
		T & T & T & T & \textbf{T} & T & T & \textbf{T} \\
		T & T & F & T & \textbf{T} & T & T & \textbf{T} \\
		T & F & T & F & \textbf{T} & T & T & \textbf{T} \\
		T & F & F & F & \textbf{F} & T & F & \textbf{F} \\
		F & T & T & F & \textbf{T} & T & T & \textbf{T} \\
		F & T & F & F & \textbf{F} & F & T & \textbf{F} \\
		F & F & T & F & \textbf{T} & T & T & \textbf{T} \\
		F & F & F & F & \textbf{F} & F & F & \textbf{F} \\ \hline\hline
	\end{tabular} 
\end{center}

$\alpha$,$\beta$,$\gamma$それぞれの真偽がいずれの場合であっても,$(\alpha \wedge \beta) \vee \gamma\ (\text{i.e.,}\ \mathscr{P}_2)$と$(\alpha \vee \gamma) \wedge (\beta \vee \gamma)\ (\text{i.e.,}\ \mathscr{P}_5)$は真偽が一致するため,この2つの命題は等価である.


\shoumon{(2) - 背理法}

$\mathscr{P}_1 \equiv \neg\beta$;\quad
$\mathscr{P}_2 \equiv \alpha \wedge \mathscr{P}_1$;\quad
$\mathscr{P}_3 \equiv (\mathscr{P}_2 \Longrightarrow \varphi)$;\quad
$\mathscr{P}_4 \equiv (\alpha \Longrightarrow \beta)$.

\begin{center}
	\begin{tabular}{ccc|cccc} \hline\hline
		$\alpha$ & $\beta$ & $\varphi$ & $\mathscr{P}_1$ & $\mathscr{P}_2$ & $\mathscr{P}_3$ & $\mathscr{P}_4$ \\ \hline
		T & T & F & F & F & \textbf{T} & \textbf{T} \\
		T & F & F & T & T & \textbf{F} & \textbf{F} \\
		F & T & F & F & F & \textbf{T} & \textbf{T} \\
		F & F & F & T & F & \textbf{T} & \textbf{T} \\ \hline\hline
	\end{tabular} 
\end{center}

$\alpha$,$\beta$,$\gamma$それぞれの真偽がいずれの場合であっても,$(\alpha \wedge \neg\beta) \Longrightarrow \varphi\ (\text{i.e.,}\ \mathscr{P}_3)$と$\alpha \Longrightarrow \beta\ (\text{i.e.,}\ \mathscr{P}_4)$は真偽が一致するため,この2つの命題は等価である.


\newpage
%%%%%%%%%%%%%%%%%%%%%%%%%%%%%%%%%%%%%%%%%%%%%%%%%%%%%%
\section{}

\begin{shadebox}
	\begin{center}
		\texttt{02\_1.pdf}\\
		\texttt{02\_2.pdf}
	\end{center}
\end{shadebox}
%\vspace{5mm}

\shoumon{(a)}

\begin{align*}
	&\neg
	\big[
	(\forall \varepsilon > 0)
	(\exists N \in \mathbb{N})
	(\forall n, \forall m \in \mathbb{N})
	\ 
	(n, m \geq N\ 
	\Longrightarrow\ 
	|a_n-a_m|<\varepsilon)
	\big] \\
	\equiv\ &
	(\exists \varepsilon >0)
	\neg
	\big[
	(\exists N \in \mathbb{N})
	(\forall n, \forall m \in \mathbb{N})
	\ 
	(n, m \geq N\ 
	\Longrightarrow\ 
	|a_n-a_m|<\varepsilon)
	\big] \\
	\equiv\ &
	(\exists \varepsilon >0)
	(\forall N \in \mathbb{N})
	\neg
	\big[
	(\forall n, \forall m \in \mathbb{N})
	\ 
	(n, m \geq N\ 
	\Longrightarrow\ 
	|a_n-a_m|<\varepsilon)
	\big] \\
	\equiv\ &
	(\exists \varepsilon >0)
	(\forall N \in \mathbb{N})
	(\exists n, \exists m \in \mathbb{N})
	\ 
	\neg 
	(n, m \geq N\ 
	\Longrightarrow\ 
	|a_n-a_m|<\varepsilon) \\
	\equiv\ &
	\uwave{
	(\exists \varepsilon >0)
	(\forall N \in \mathbb{N})
	(\exists n, \exists m \in \mathbb{N})
	\ 
	[(n, m \geq N) \wedge (|a_n-a_m| \geq \varepsilon)]
	}.
\end{align*}
ただし最後の変形で
\begin{equation*}
	\neg(\alpha\Longrightarrow\beta)
	\equiv
	\neg[(\neg\alpha) \vee \beta]
	\equiv
	[\neg(\neg\alpha) \wedge (\neg\beta)]
	\equiv
	[\alpha \wedge (\neg\beta)]
\end{equation*}
なる関係を用いた.


\shoumon{(b)}

(1.2)
\begin{align*}
	&\neg
	\big[
	(\forall x \in \mathbb{R})
	(\forall \varepsilon > 0)
	(\exists \delta > 0)
	(\forall y \in \mathbb{R})
	\ 
	(|x-y| < \delta\ 
	\Longrightarrow\ 
	|f(x)-f(y)| < \varepsilon)
	\big] \\
	\equiv\ &
	(\exists x \in \mathbb{R})
	(\exists \varepsilon > 0)
	\neg
	\big[
	(\exists \delta > 0)
	(\forall y \in \mathbb{R})
	\ 
	(|x-y| < \delta\ 
	\Longrightarrow\ 
	|f(x)-f(y)| < \varepsilon)
	\big] \\
	\equiv\ &
	(\exists x \in \mathbb{R})
	(\exists \varepsilon > 0)
	(\forall \delta > 0)
	\neg
	\big[
	(\forall y \in \mathbb{R})
	\ 
	(|x-y| < \delta\ 
	\Longrightarrow\ 
	|f(x)-f(y)| < \varepsilon)
	\big] \\
	\equiv\ &
	(\exists x \in \mathbb{R})
	(\exists \varepsilon > 0)
	(\forall \delta > 0)
	(\exists y \in \mathbb{R})
	\ 
	\neg
	(|x-y| < \delta\ 
	\Longrightarrow\ 
	|f(x)-f(y)| < \varepsilon)
	\big] \\
	\equiv\ &
	\uwave{
	(\exists x \in \mathbb{R})
	(\exists \varepsilon > 0)
	(\forall \delta > 0)
	(\exists y \in \mathbb{R})
	\ 
	[(|x-y| < \delta)
	\wedge
	(|f(x)-f(y)| \geq \varepsilon)]
	}.
\end{align*}


(1.3)
\begin{align}
	&\neg
	\big[
	(\forall \varepsilon > 0)
	(\exists \delta > 0)
	(\forall x \in \mathbb{R})
	(\forall y \in \mathbb{R})
	\ 
	(|x-y| < \delta\ 
	\Longrightarrow\ 
	|f(x)-f(y)| < \varepsilon)
	\big] \notag \\
	\equiv\ &
	(\exists \varepsilon > 0)
	\neg
	\big[
	(\exists \delta > 0)
	(\forall x \in \mathbb{R})
	(\forall y \in \mathbb{R})
	\ 
	(|x-y| < \delta\ 
	\Longrightarrow\ 
	|f(x)-f(y)| < \varepsilon)
	\big] \notag \\
	\equiv\ &
	(\exists \varepsilon > 0)
	(\forall \delta > 0)
	\neg
	\big[
	(\forall x \in \mathbb{R})
	(\forall y \in \mathbb{R})
	\ 
	(|x-y| < \delta\ 
	\Longrightarrow\ 
	|f(x)-f(y)| < \varepsilon)
	\big] \notag \\
	\equiv\ &
	(\exists \varepsilon > 0)
	(\forall \delta > 0)
	(\exists x \in \mathbb{R})
	(\exists y \in \mathbb{R})
	\ 
	\neg
	(|x-y| < \delta\ 
	\Longrightarrow\ 
	|f(x)-f(y)| < \varepsilon)
	\big] \notag \\
	\equiv\ &
	\uwave{
	(\exists \varepsilon > 0)
	(\forall \delta > 0)
	(\exists x \in \mathbb{R})
	(\exists y \in \mathbb{R})
	\ 
	[(|x-y| < \delta)
	\wedge
	(|f(x)-f(y)| \geq \varepsilon)]
	}.
	\label{eq1}
\end{align}


\shoumon{(c)}

たとえば
\begin{equation*}
	f(\xi) = \xi^2
\end{equation*}
は連続ではあるが,一様連続ではない.

($\because$)

恣意的に$\varepsilon=1/2$とする.任意の$\delta>0$を用いて,$x \geq 1/(2\delta)$,$y=x+\delta/2$とする.
このとき,
\begin{equation*}
	|x-y| =\left|x-\left(x+\frac{\delta}{2}\right)\right| = \frac{\delta}{2}
	< \delta
\end{equation*}
を満たす.
また,
\begin{equation*}
	|f(x)-f(y)|
	= \left|x^2-\left(x+\frac{\delta}{2}\right)^2\right| 
	= x\delta + \frac{\delta^2}{4} 
	\geq \frac{1}{2} + \frac{\delta^2}{4} 
	\geq \frac{1}{2}
	= \varepsilon
\end{equation*}
も同時に満たす.
これで命題(\ref{eq1})が示されたため,$f(\xi)=\xi^2$が一様連続ではないことが分かった. \qed

\newpage
%%%%%%%%%%%%%%%%%%%%%%%%%%%%%%%%%%%%%%%%%%%%%%%%%%%%%%
\section{}

\begin{shadebox}
	\begin{center}
		\texttt{03.pdf}
	\end{center}
\end{shadebox}
%\vspace{5mm}

\shoumon{分配律 1}

$(A \cap B) \cup C \subset (A \cup C) \cap (B \cup C)$
および
$(A \cap B) \cup C \supset (A \cup C) \cap (B \cup C)$
が成立するとき,
$(A \cap B) \cup C = (A \cup C) \cap (B \cup C)$
である.
以下でそれぞれを順番に示す.

($\subset$)

$\forall x \in (A \cap B) \cup C$について,
\begin{equation*}
	x \in A \cap B,
	\quad \text{または} \quad
	x \in C
\end{equation*}
である.
いま,
\begin{equation*}
	A \cap B \ (\subset A) \subset A \cup C, \quad
	A \cap B \ (\subset B) \subset B \cup C
\end{equation*}
であるから,
\begin{equation*}
	x \in A \cap B
	\quad \Longrightarrow \quad
	x \in (A \cup C) \cap (B \cup C).
\end{equation*}
また,
\begin{equation*}
	C \subset A \cup C, \quad
	C \subset B \cup C
\end{equation*}
であるから,
\begin{equation*}
	x \in C
	\quad \Longrightarrow \quad
	x \in (A \cup C) \cap (B \cup C).
\end{equation*}
したがって,
$x \in A \cap B$ と $x \in C$
いずれの場合であっても
$x \in (A \cup C) \cap (B \cup C)$
が成立し,

$(A \cap B) \cup C \subset (A \cup C) \cap (B \cup C)$
が示された.

($\supset$)

$\forall x \in (A \cup C) \cap (B \cup C)$について,
\begin{equation*}
	x \in A \cup C,
	\quad \text{かつ} \quad
	x \in B \cup C
\end{equation*}
である.
いま,
\begin{equation*}
	x \in C,
	\quad \text{または} \quad
	x \not\in C
\end{equation*}
が自明である.
$x \not\in C$のとき,
\begin{equation*}
	x \in A \cup C
	\ \text{かつ}\ 
	x \not\in C
	\quad \Longrightarrow \quad
	x \in A,
\end{equation*}
および,
\begin{equation*}
	x \in B \cup C
	\ \text{かつ}\ 
	x \not\in C
	\quad \Longrightarrow \quad
	x \in B
\end{equation*}
が同時に成り立つ.
すなわち,$x \not\in C$のとき$x \in A \cap B$である.
したがって,
\begin{equation*}
	x \in C,
	\quad \text{または} \quad
	x \in A \cap B
\end{equation*}
であり,
$x \in (A \cap B) \cup C$
が成立し,
$(A \cap B) \cup C \supset (A \cup C) \cap (B \cup C)$
が示された.
\qed


\shoumon{分配律 2}

$(A \cup B) \cap C \subset (A \cap C) \cup (B \cap C)$
および
$(A \cup B) \cap C \supset (A \cap C) \cup (B \cap C)$
が成立するとき,
$(A \cup B) \cap C = (A \cap C) \cup (B \cap C)$
である.
以下でそれぞれを順番に示す.

($\subset$)

$\forall x \in (A \cup B) \cap C$について,
\begin{equation*}
	x \in A \cup B,
	\quad \text{かつ} \quad
	x \in C
\end{equation*}
である.
いま,$x \in A \cup B$より
\begin{equation*}
	x \in A,
	\quad \text{または} \quad
	x \in B
\end{equation*}
であるから,
\begin{equation*}
	x \in A,
	\quad \text{かつ} \quad
	x \in C,
\end{equation*}
すなわち$x \in A \cap C$,または,
\begin{equation*}
	x \in B,
	\quad \text{かつ} \quad
	x \in C,
\end{equation*}
すなわち$x \in B \cap C$が成立する.
したがって
$x \in (A \cap C) \cup (B \cap C)$
が成立し,
$(A \cup B) \cap C \subset (A \cap C) \cup (B \cap C)$
が示された.

($\supset$)

$\forall x \in (A \cap C) \cup (B \cap C)$について,
\begin{equation*}
	x \in A \cap C,
	\quad \text{または} \quad
	x \in B \cap C
\end{equation*}
であり,言い換えれば,
\begin{equation*}
	x \in A,
	\quad \text{かつ} \quad
	x \in C
\end{equation*}
または,
\begin{equation*}
	x \in B,
	\quad \text{かつ} \quad
	x \in C
\end{equation*}
である.
いま,
\begin{equation*}
	A \subset A \cup B,
	\quad
	B \subset A \cup B
\end{equation*}
であるから,$x \in A \cap C$ならば,
\begin{equation*}
	x \in A \cup B
	\quad \text{かつ} \quad
	x \in C,
\end{equation*}
すなわち,$x \in (A \cup B) \cap C$,
$x \in B \cap C$ならば,
\begin{equation*}
	x \in A \cup B
	\quad \text{かつ} \quad
	x \in C.
\end{equation*}
すなわち,$x \in (A \cup B) \cap C$.
したがって,
$x \in A \cap C$と$x \in B \cap C$
いずれの場合であっても$x \in (A \cup B) \cap C$
が成立し,
$(A \cup B) \cap C \supset (A \cap C) \cup (B \cap C)$
が示された.
\qed


\shoumon{一般化 De Morgen 律 1}

$C \setminus (A \cap B) \subset (C \setminus A) \cup (C \setminus B)$
および
$C \setminus (A \cap B) \supset (C \setminus A) \cup (C \setminus B)$
が成立するとき,
$C \setminus (A \cap B) = (C \setminus A) \cup (C \setminus B)$
である.
以下でそれぞれを順番に示す.

($\subset$)

$\forall x \in C \setminus (A \cap B)$について,
\begin{equation*}
	x \in  C,
	\quad \text{かつ} \quad
	x \not\in A \cap B
\end{equation*}
である.
$x \not\in A \cap B$より,
\begin{equation*}
	x \not\in A,
	\quad \text{または} \quad
	x \not\in B
\end{equation*}
であるから,
\begin{equation*}
	x \in C,
	\quad \text{かつ} \quad
	x \not\in A,
\end{equation*}
すなわち$x \in C \setminus A$,または,
\begin{equation*}
	x \in C,
	\quad \text{かつ} \quad
	x \not\in B,
\end{equation*}
すなわち$x \in C \setminus B$が成立する.
したがって
$x \in (C \setminus A) \cup (C \setminus B)$
が成立し,
$C \setminus (A \cap B) \subset (C \setminus A) \cup (C \setminus B)$
が示された.


($\supset$)

$\forall x \in (C \setminus A) \cup (C \setminus B)$について,
\begin{equation*}
	x \in  C \setminus A,
	\quad \text{または} \quad
	x \in C \setminus B
\end{equation*}
であり,言い換えれば,
\begin{equation*}
	x \in  C,
	\quad \text{かつ} \quad
	x \not\in A
\end{equation*}
または,
\begin{equation*}
	x \in  C,
	\quad \text{かつ} \quad
	x \not\in B
\end{equation*}
である.
$A \supset A \cap B$より,
\begin{equation*}
	x \not\in A
	\quad \Longrightarrow \quad
	x \not\in A \cap B.
\end{equation*}
同様にして,
\begin{equation*}
	x \not\in B
	\quad \Longrightarrow \quad
	x \not\in A \cap B.
\end{equation*}
したがって,
$x \in  C \setminus A$と$x \in C \setminus B$
いずれの場合であっても
\begin{equation*}
	x \in  C,
	\quad \text{かつ} \quad
	x \not\in A \cap B,
\end{equation*}
すなわち$x \in C \setminus (A \cap B)$が成立し,
$C \setminus (A \cap B) \supset (C \setminus A) \cup (C \setminus B)$
が示された.
\qed


\shoumon{一般化 De Morgen 律 2}

$C \setminus (A \cup B) \subset (C \setminus A) \cap (C \setminus B)$
および
$C \setminus (A \cup B) \supset (C \setminus A) \cap (C \setminus B)$
が成立するとき,
$C \setminus (A \cup B) = (C \setminus A) \cap (C \setminus B)$
である.
以下でそれぞれを順番に示す.


($\subset$)

$\forall x \in C \setminus (A \cup B)$について,
\begin{equation*}
	x \in  C,
	\quad \text{かつ} \quad
	x \not\in A \cup B
\end{equation*}
である.
$x \not\in A \cup B$より,
\begin{equation*}
	x \not\in A,
	\quad \text{かつ} \quad
	x \not\in B
\end{equation*}
であるから,
\begin{equation*}
	x \in C,
	\quad \text{かつ} \quad
	x \not\in A,
\end{equation*}
すなわち$x \in C \setminus A$,かつ,
\begin{equation*}
	x \in C,
	\quad \text{かつ} \quad
	x \not\in B,
\end{equation*}
すなわち$x \in C \setminus B$が成立する.
したがって
$x \in (C \setminus A) \cap (C \setminus B)$
が成立し,
$C \setminus (A \cup B) \subset (C \setminus A) \cap (C \setminus B)$
が示された.


($\supset$)

$\forall x \in (C \setminus A) \cap (C \setminus B)$について,
\begin{equation*}
	x \in  C \setminus A,
	\quad \text{かつ} \quad
	x \in C \setminus B
\end{equation*}
であり,言い換えれば,
\begin{equation*}
	x \in  C,
	\quad \text{かつ} \quad
	x \not\in A
\end{equation*}
かつ,
\begin{equation*}
	x \in  C,
	\quad \text{かつ} \quad
	x \not\in B
\end{equation*}
である.
$x \not\in A$かつ$x \not\in B$より
\begin{equation*}
	x \not\in A \cup B.
\end{equation*}
したがって,
$x \in C \setminus (A \cup B)$が成立し,
$C \setminus (A \cup B) \supset (C \setminus A) \cap (C \setminus B)$
が示された.
\qed

\newpage
%%%%%%%%%%%%%%%%%%%%%%%%%%%%%%%%%%%%%%%%%%%%%%%%%%%%%%
\section{}

\begin{shadebox}
	\begin{center}
		\texttt{04.pdf}
	\end{center}
\end{shadebox}
\vspace{5mm}

(1)--(4)の集合をそれぞれ,$A_1$,$A_2$,$A_3$,$A_4$とする.

\shoumon{(1)}

\uwave{$\sup A_1 = \pi$}.$\pi \not\in A_1$より\uwave{$\max A_1$は存在しない}.

\uwave{$\inf A_1 = \min A_1 = 0$}\ $\in A_1$.

\shoumon{(2)}

$\sup A_2 = u > 0$が存在すると仮定.
$A_2$の定義より$(-1)^\nu \nu = u$を満たす$\nu \in \mathbb{N}$が存在する.
また,$u > 0$なる仮定より$\nu$は偶数である.
このとき,$\nu + 2 \in \mathbb{N}$より,
$(-1)^{\nu + 2} (\nu + 2) \in A_2$.
ここで,
$(-1)^{\nu + 2} (\nu + 2) = (-1)^{\nu} (\nu + 2) = u + (-1)^{\nu}2 > u$
より,$u$が$A_2$の上限であることに矛盾する.
したがって\uwave{$A_2$に上限は存在しない}.
\uwave{$\max A_2$も存在しない}.

同様にして,\uwave{$\inf A_2$,$\min A_2$も存在しない}.


\shoumon{(3)}

\uwave{$\sup A_3 = \max A_3 = \ln 1 = 0$}\ $\in A_3$.

$\ln x \xrightarrow{x \rightarrow +0} -\infty$であるから,
\uwave{$\inf A_3$,$\min A_3$は存在しない}.


\shoumon{(4)}

$e^x$ は $(e^x)' = e^x > 0 \ (\forall x \in \mathbb{R})$ より単調増加関数である.

$\ln x \xrightarrow{x \rightarrow \infty} \infty$であるから,
\uwave{$\sup A_4$,$\max A_4$は存在しない}.

$e^x$の単調性と$\ln x \xrightarrow{x \rightarrow -\infty} 0$,
および
$e^x > 0 \ (\forall x \in \mathbb{R})$より
\uwave{$\inf A_4=0$}.
しかし,$0 \not\in A_4$ゆえ,\uwave{$\min A_4$は存在しない}.

\newpage
%%%%%%%%%%%%%%%%%%%%%%%%%%%%%%%%%%%%%%%%%%%%%%%%%%%%%%
\section{}

\begin{shadebox}
	\begin{center}
		\texttt{05.pdf}
	\end{center}
\end{shadebox}
%\vspace{5mm}
%
%\begin{irohako}[blue]{収束列の定義}
%	数列$\{a_n\}_{n \in \mathbb{N}}$が収束列であるとは,
%	\begin{equation*}
%		(\exists \alpha \in \mathbb{R})
%		(\forall \epsilon > 0)
%		(\exists N \in \mathbb{N})
%		(\forall n \in \mathbb{N})
%		\big[
%		n \geq N
%		\ \Longrightarrow\ 
%		|a_n -\alpha| < \epsilon
%		\big]
%	\end{equation*}
%	が成り立つこと.
%\end{irohako}
%
%\begin{irohako}[blue]{Caucy列の定義}
%	数列$\{a_n\}_{n \in \mathbb{N}}$がCaucy列であるとは,
%	\begin{equation*}
%	(\forall \epsilon > 0)
%	(\exists N \in \mathbb{N})
%	(\forall m, \forall n \in \mathbb{N})
%	\big[
%	(m \geq N)
%	\wedge
%	(n \geq N)
%	\ \Longrightarrow\ 
%	|a_m -a_n| < \epsilon
%	\big]
%	\end{equation*}
%	が成り立つこと.
%\end{irohako}
%
%\begin{irohako}[magenta]{命題 1}
%	収束列はCaucy列である.
%\end{irohako}
%
%($\because$)
%
%$\{a_n\}_{n \in \mathbb{N}}$が$a_n \longrightarrow \alpha$ なる収束列であるとする.
%以下では任意の$\epsilon >0$を固定して考える.
%\textsf{収束列の定義}から,$\epsilon/2$に対してある$N$が存在して,
%\begin{align*}
%	n \geq N \ \Longrightarrow\ |a_n-\alpha| < \frac{\epsilon}{2},
%	\quad
%	m \geq N \ \Longrightarrow\ |a_m-\alpha| < \frac{\epsilon}{2}
%\end{align*}
%が成立.
%このとき,任意の$m, n \geq N$に対して,
%\begin{equation*}
%	|a_m - a_n| \underset{\text{\scriptsize (三角不等式)}}{\leq} |a_n-\alpha| + |a_m-\alpha| \leq \epsilon
%\end{equation*}
%が成立.
%\textsf{収束列の定義}より,数列$\{a_n\}_{n \in \mathbb{N}}$はCaucy列でもある.
%以上より\textsf{命題1}は示された.
%\qed
%
%また,以下も成立する.
%\begin{irohako}[magenta]{命題 1 の対偶}
%	Caucy列でないならば,収束列ではない.
%\end{irohako}
%ここで,\textsf{問題 1.2 (a)}より以下が分かっている.
%\begin{irohako}[magenta]{Caucy列の定義の否定}
%	数列$\{a_n\}_{n \in \mathbb{N}}$がCaucy列でないとは,
%	\begin{equation*}
%		(\exists \epsilon > 0)
%		(\forall N \in \mathbb{N})
%		(\exists m, \exists n \in \mathbb{N})
%		\big[
%		(m, n \geq N)
%		\wedge
%		(|a_m -a_n| \geq \epsilon)
%		\big]
%	\end{equation*}
%	が成り立つこと.
%\end{irohako}

\shoumon{(1)}

$0 < a < 1$の場合,数列$\{a_n\}_{n \in \mathbb{N}}$は\uwave{0}に収束する.
%し,
%\begin{equation*}
%	\lim_{n\rightarrow \infty} a_n = \uwave{0}.
%\end{equation*}

$a = 1$の場合,数列$\{a_n\}_{n \in \mathbb{N}}$は収束し,
\begin{equation*}
\lim_{n\rightarrow \infty} a_n = \lim_{n\rightarrow \infty} 1^n = \uwave{1}.
\end{equation*}

$a > 1$の場合,数列$\{a_n\}_{n \in \mathbb{N}}$は\uwave{収束しない}
%\footnote{
%	数列がCaucy列でないことを示すことによって,収束列ではないことを示す.\\
%	$0 < \epsilon < \infty$を仮定する.
%	任意の自然数$N$に対して$m=N$,$(m<)\ n \longrightarrow \infty$を仮定すると,
%	$|a_m-a_n| = \infty \geq \epsilon$.
%	したがってこの数列はCaucy列ではなく,収束しない.
%	\qed
%}
.
なんとなればすなわち,任意の$a > 1$に対して
\begin{equation*}
	\lim_{n\rightarrow \infty} a_n = \lim_{n\rightarrow \infty} a^n = \infty.
\end{equation*}


\shoumon{(2)}

$-\infty < a < 0$の場合,数列$\{a_n\}_{n \in \mathbb{N}}$は\uwave{0}に収束する.
%し,
%\begin{equation*}
%	\lim_{n\rightarrow \infty} a_n = \uwave{0}.
%\end{equation*}

$a = 0$の場合,数列$\{a_n\}_{n \in \mathbb{N}}$は収束し,
\begin{equation*}
	\lim_{n\rightarrow \infty} a_n = \lim_{n\rightarrow \infty} n^0 = \uwave{1}.
\end{equation*}

$0 < a < \infty$の場合,数列$\{a_n\}_{n \in \mathbb{N}}$は\uwave{収束しない}.
なんとなればすなわち,任意の$a > 0$に対して
\begin{equation*}
	\lim_{n\rightarrow \infty} a_n = \lim_{n\rightarrow \infty} n^a = \infty.
\end{equation*}


\shoumon{(3)}

数列$\{a_n\}_{n \in \mathbb{N}}$は\uwave{0}に収束する.
なんとなればすなわち,$n \longrightarrow \infty$の極限において
$a^n \gg n^b \ (\text{for}\ \forall a>1,\ \forall b>0)$であるから,
\begin{equation*}
	\lim_{n\rightarrow \infty} a_n = 0.
\end{equation*}


\shoumon{(4)}

数列$\{a_n\}_{n \in \mathbb{N}}$は\uwave{1}に収束する.
なんとなればすなわち,$n \longrightarrow \infty$の極限において
$1/n \longrightarrow 0$であるから,
\begin{equation*}
	\lim_{n\rightarrow \infty} a_n = a^0 = 1.
\end{equation*}


\newpage
%%%%%%%%%%%%%%%%%%%%%%%%%%%%%%%%%%%%%%%%%%%%%%%%%%%%%%
\section{}

\begin{shadebox}
	\begin{center}
		\texttt{06.pdf}
	\end{center}
\end{shadebox}
\vspace{5mm}

式(1.4)を変形することにより,
\begin{equation*}
a_n = \frac{1}{\sqrt{n+1}+\sqrt{n}}
\end{equation*}
が得られる.
$\forall m > \forall n$に対して,
\begin{align*}
	a_m - a_n
	& = \frac{1}{\sqrt{m+1}+\sqrt{m}} - \frac{1}{\sqrt{n+1}+\sqrt{n}} \\
	& = -\frac{\sqrt{m+1}-\sqrt{n+1}}{(\sqrt{m+1}+\sqrt{m})(\sqrt{n+1}+\sqrt{n})} \\
	& < 0
\end{align*}
より,この数列は単調減少である.
また,
\begin{equation*}
	\lim_{n\rightarrow \infty} a_n
	= \lim_{n\rightarrow \infty} \frac{1}{\sqrt{n+1}+\sqrt{n}}
	= 0
\end{equation*}
より,この数列は0に収束する.
\qed

\newpage
%%%%%%%%%%%%%%%%%%%%%%%%%%%%%%%%%%%%%%%%%%%%%%%%%%%%%%
\section{}

\begin{shadebox}
	\begin{center}
		\texttt{07.pdf}
	\end{center}
\end{shadebox}
%\vspace{5mm}

\shoumon{(1)}

$n$個の$(n+1)/n$と1について,$\text{(相加平均)} \geq \text{(相乗平均)}$であるから,
\begin{alignat*}{2}
	&&\frac{n\cdot\frac{n+1}{n}+1}{n+1}
	&\geq
	\left[\left( \frac{n+1}{n} \right)^n \cdot 1\right]^{\frac{1}{n+1}} \\
	\Longrightarrow\quad
	&&\frac{n+2}{n+1}
	&\geq
	\left( \frac{n+1}{n} \right)^{\frac{n}{n+1}} \\
	\Longrightarrow\quad
	&&\left(\frac{n+2}{n+1}\right)^{{n+1}}
	&\geq
	\left( \frac{n+1}{n} \right)^{n} \\
	\Longrightarrow\quad
	&&\left(1+\frac{1}{n+1}\right)^{n+1}
	&\geq
	\left(1+\frac{1}{n}\right)^{n} \\
	\Longrightarrow\quad
	&&a_{n+1}
	&\geq
	a_n.
\end{alignat*}
この数列は単調非減少である. \qed

\newpage
\shoumon{(2)}

二項定理
\begin{equation*}
	(\xi + \eta)^\nu 
	= \sum_{k=0}^\nu
		\binom{\nu}{k}
		\xi^{\nu-k}\eta^k
	= \sum_{k=0}^\nu
		\frac{\nu!}{k!(\nu-k)!}
		\xi^{\nu-k}\eta^k
\end{equation*}
を用いて,
\begin{align*}
	a_n 
	&=
	\left(1+\frac{1}{n}\right)^{n} \\
	&=
	\sum_{k=0}^n
		\frac{n!}{k!(n-k)!}
		\left(\frac{1}{n}\right)^{k} \\
	&=
	\sum_{k=0}^n
		\frac{1}{k!}
		\left(
			\frac{n}{n}
			\cdot
			\frac{n-1}{n}
			\cdot
			\frac{n-2}{n}
			\cdot \cdots \cdot
			\frac{n-(k-1)}{n}
		\right) \\
	&=
	\sum_{k=0}^n
		\frac{1}{k!}
		\left[
			1
			\cdot
			\left(1-\frac{1}{n}\right)
			\cdot
			\left(1-\frac{2}{n}\right)
			\cdot \cdots \cdot
			\left(1-\frac{k-1}{n}\right)
		\right] \\
	&\leq
	\sum_{k=0}^n \frac{1}{k!} \\
	&=
	1+1+\frac{1}{2}+\frac{1}{3\cdot 2}+\cdots +\frac{1}{n!} \\
	&\leq
	1+ \sum_{i=0}^{n} \frac{1}{2^i} \\
	&\leq
	1+ \sum_{i=0}^{\infty} \frac{1}{2^i} \\
	&= 1 + \frac{1}{1-1/2} \\
	&= 3.
\end{align*}
$a_n\leq 3$よりこの数列は上に有界であり,3が上界に含まれる. \qed

\newpage
%%%%%%%%%%%%%%%%%%%%%%%%%%%%%%%%%%%%%%%%%%%%%%%%%%%%%%
\section{}

\begin{shadebox}
	\begin{center}
		\texttt{08.pdf}
	\end{center}
\end{shadebox}
\vspace{5mm}

\begin{irohako}[ForestGreen]{Integral test{\footnotemark}}
	Let $f$ be a function defined for all numbers $\geq 1$.
	Assume that $f(x) \geq 0$ for all $x$, that $f$ is decresing, and that
	\begin{equation*}
		\lim_{U\rightarrow\infty}
		\int_1^U f(x) \bibun x
	\end{equation*}
	exits.
	Then the series
	\begin{equation*}
		\sum_{n=1}^\infty f(n)
	\end{equation*}
	converges.
\end{irohako}
\footnotetext{Lang, S. (1997), \textsl{Undergraduate Analysis}, Springer-Verlag.}

\begin{irohako}[blue]{d'Alembertの収束判定{\footnotemark}}
	$\sum a_n$は正項級数とする.有限個の$n$を除き
	\begin{equation*}
		\frac{a_{n+1}}{a_n} \leq r
	\end{equation*}
	となる$r>1$が存在すれば,$\sum a_n$は収束する.
	なお,$r=1$のとき,級数によっては収束することもあれば発散することもある.
\end{irohako}
\footnotetext{三宅 敏恒 (1992), 『入門微分積分』, 培風館.}

\newpage
\shoumon{(1)}

$f(x)=1/x$とする.この$f$は$f(x)\geq 0$かつ減少関数である.
このとき,
\begin{equation*}
	\lim_{U\rightarrow\infty}
	\int_1^U \frac{1}{x} \bibun x
	=
	\lim_{U\rightarrow\infty}
	\ln (U)
	=
	\infty
\end{equation*}
ゆえ,\textsf{Integral test}より,この級数は収束しない.


\shoumon{(2)}

$f(x)=1/(x^2)$とする.この$f$は$f(x)\geq 0$かつ減少関数である.
このとき,
\begin{equation*}
	\lim_{U\rightarrow\infty}
	\int_1^U \frac{1}{x^2} \bibun x
	=
	\lim_{U\rightarrow\infty}
	\left(1-\frac{1}{U}\right)
	=
	1
\end{equation*}
ゆえ,\textsf{Integral test}より,この級数は収束する.


\shoumon{(3)}

$a_n = 1/n!$とおく.
このとき,
\begin{equation*}
	\lim_{n\rightarrow\infty} \frac{a_{n+1}}{a_n}
	=
	\lim_{n\rightarrow\infty} \frac{1/(n+1)!}{1/n!}
	=
	\lim_{n\rightarrow\infty} \frac{1}{n+1}
	= 0 <1.
\end{equation*}
ゆえ,\textsf{d'Alembertの収束判定}より,この級数は収束する.

\newpage
%%%%%%%%%%%%%%%%%%%%%%%%%%%%%%%%%%%%%%%%%%%%%%%%%%%%%%
\section{}

\begin{shadebox}
	\begin{center}
		\texttt{09.pdf}
	\end{center}
\end{shadebox}
\vspace{5mm}

以下では数列(の一般項)を$a_n$で表す.

\shoumon{(1)}

$n(k)=2k-1 \ (k\in\mathbb{N})$とすると,
\begin{equation*}
	a_{n(k)}
	=
	(2k-1)\cos\left(\frac{\pi}{2}(2k-1)\right)
	=0
\end{equation*}
より,部分列$\{a_{n(k)}\}_{k\in\mathbb{N}}$は\uwave{0}に収束する.

$n(k)=4k-2 \ (k\in\mathbb{N})$とすると,
\begin{equation*}
	a_{n(k)}
	=
	(4k-2)\cos\left(\frac{\pi}{2}(4k-2)\right)
	=-(4k-2)
\end{equation*}
より,部分列$\{a_{n(k)}\}_{k\in\mathbb{N}}$は$-\infty$に発散する.


\shoumon{(2)}

$n(k)=2k-1 \ (k\in\mathbb{N})$とすると,
\begin{equation*}
	a_{n(k)}
	=
	(2k-1)[1+(-1)^{2k-1}]
	=0
\end{equation*}
より,部分列$\{a_{n(k)}\}_{k\in\mathbb{N}}$は\uwave{0}に収束する.

$n(k)=2k \ (k\in\mathbb{N})$とすると,
\begin{equation*}
	a_{n(k)}
	=
	2k[1+(-1)^{2k}]
	=4k
\end{equation*}
より,部分列$\{a_{n(k)}\}_{k\in\mathbb{N}}$は$+\infty$に発散する.


\shoumon{(3)}

$n(k)=2k-1 \ (k\in\mathbb{N})$とすると,
\begin{equation*}
	a_{n(k)}
	=
	(-1)^{2k-1}\frac{2k-1}{2k}
	=
	-\frac{2k-1}{2k}
\end{equation*}
より,部分列$\{a_{n(k)}\}_{k\in\mathbb{N}}$は$-1$に収束する.

$n(k)=2k \ (k\in\mathbb{N})$とすると,
\begin{equation*}
a_{n(k)}
=
(-1)^{2k}\frac{2k}{2k+1}
=
\frac{2k}{2k+1}
\end{equation*}
より,部分列$\{a_{n(k)}\}_{k\in\mathbb{N}}$は1に収束する.


\newpage
%%%%%%%%%%%%%%%%%%%%%%%%%%%%%%%%%%%%%%%%%%%%%%%%%%%%%%
\section{}

\begin{shadebox}
	\begin{center}
		\texttt{10.pdf}
	\end{center}
\end{shadebox}
\vspace{5mm}

%\begin{equation*}
%	f(x) = |x| =
%	\left\{
%	\begin{array}{ll}
%		x & \text{if}\ x \geq 0; \\
%		-x & \text{if}\ x < 0.
%	\end{array}
%	\right.
%\end{equation*}

\begin{itemize}
	\item[(1)]
		\begin{equation*}
			f(A \cap B)
			= f\big((-1, 1)\big)
			= \uwave{[0, 1)}.
		\end{equation*}
	\item[(2)]
		\begin{equation*}
			f(A) \cap f(B)
			= f\big((-2, 1)\big) \cap f\big((-1, 2]\big)
			= [0, 2) \cap [0, 2]
			= \uwave{[0, 2)}.
		\end{equation*}
	\item[(3)]
		\begin{equation*}
			f^{-1}\big( f(A) \big)
			= f^{-1}\Big(f\big( (-2, 1) \big)\Big)
			= f^{-1}\big( [0, 2) \big)
			= \uwave{(-2, 2)}.
		\end{equation*}
	\item[(4)]
		\begin{equation*}
			f\big( f^{-1}(C) \big)
			= f\Big(f^{-1}\big( [0, 3) \big)\Big)
			= f\big( (-3, 3) \big)
			= \uwave{(-3, 3)}.
		\end{equation*}
\end{itemize}

\newpage
%%%%%%%%%%%%%%%%%%%%%%%%%%%%%%%%%%%%%%%%%%%%%%%%%%%%%%
\section{}

\begin{shadebox}
	\begin{center}
		\texttt{11.pdf}
	\end{center}
\end{shadebox}
%\vspace{5mm}

\shoumon{(1)}

$\forall y \in Y$について,
$f(x)=y$なる$x\in X$が存在するため,\uwave{$f$は全射}.

$\forall x_1, x_2 \in X$について,
\begin{equation*}
	f(x_1) = f(x_2)
	\quad \Longrightarrow \quad
	2x_1+3 = 2x_2+3
	\quad \Longrightarrow \quad
	x_1 = x_2
\end{equation*}
が成立するため,\uwave{$f$は単射}.

結局,$f$は全単射.

$y=f(x)$を$x$について解くと逆関数$f^{-1}$が以下の通り求められる.
\begin{equation*}
	y = 2x+3
	\quad \Longrightarrow \quad
	(x =)\ \uwave{f^{-1}(y) = \frac{1}{2}y - \frac{3}{2}}.
\end{equation*}


\shoumon{(2)}

$\forall y \in Y$について,
$f(x)=y$なる$x\in X$が存在するため,\uwave{$f$は全射}.

$x_1=1 \in X$,$x_2=-1 \in X$について,$x_1 \neq x_2$かつ
\begin{equation*}
	f(x_1) = f(x_2) = 1
\end{equation*}
が成立しているため,\uwave{$f$は単射ではない}.


\shoumon{(3)}

$y=8 \in Y$について,
$f(x)=y$なる$x$は$x=3 \not\in X$であるから,\uwave{$f$は全射ではない}.

$\forall x_1, x_2 \in X$について,
\begin{equation*}
	f(x_1) = f(x_2)
	\quad \Longrightarrow \quad
	x_1^2-1 = x_2^2-1
	\quad \Longrightarrow \quad
	x_1 = x_2
\end{equation*}
が成立するため,\uwave{$f$は単射}.

\newpage
%%%%%%%%%%%%%%%%%%%%%%%%%%%%%%%%%%%%%%%%%%%%%%%%%%%%%%
\section{}

\begin{shadebox}
	\begin{center}
		\texttt{12.pdf}
	\end{center}
\end{shadebox}
\vspace{5mm}

$(f\circ g)(x) := f\big(g(x)\big)$,$(g\circ f)(x) := g\big(f(x)\big)$.

\shoumon{(1)}

$f\circ g: \mathbb{R}\longrightarrow\mathbb{R}$:
\begin{equation*}
	(f\circ g)(x) = 2(4x+5)+3
	= \uwave{8x+13}.
\end{equation*}

$g\circ f: \mathbb{R}\longrightarrow\mathbb{R}$:
\begin{equation*}
	(g\circ f)(x) = 4(2x+3)+5
	= \uwave{8x+17}.
\end{equation*}


\shoumon{(2)}

$f\circ g: \mathbb{R}_+\longrightarrow\mathbb{R}_+$:
\begin{equation*}
	(f\circ g)(x) = (\sqrt{x})^2
	= \uwave{x}.
\end{equation*}

$g\circ f: \mathbb{R}\longrightarrow\mathbb{R}$:
\begin{equation*}
	(g\circ f)(x) = \sqrt{x^2}
	= \uwave{|x|}.
\end{equation*}


\shoumon{(3)}

$g(y)=-(y+1)(y-2)$.

$f\circ g: \mathbb{R}\longrightarrow\mathbb{R}$:
\begin{equation*}
	(f\circ g)(x) = 2(-x^2+x+2)-4
	= \uwave{-2x^2+2x}.
\end{equation*}

$g\circ f: \mathbb{R}\longrightarrow\mathbb{R}$:
\begin{equation*}
	(g\circ f)(x) = -[(2x-4)+1][(2x-4)-2]
	= -(2x-3)(2x-6)
	= \uwave{-4x^2+18x-18}.
\end{equation*}

\end{document}
