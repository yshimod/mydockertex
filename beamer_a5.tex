\documentclass[uplatex,12pt,dvipdfmx,xcolor=svgnames]{beamer}
\usepackage[sourcehan]{pxchfon}
\geometry{papersize={140mm, 190mm}}

%\setbeamertemplate{navigation symbols}{}

\usepackage{bxdpx-beamer}
\usepackage{minijs}

\usepackage{atbegshi}
\AtBeginDvi{\special{pdf:tounicode EUC-UCS2}}

\usetheme{Madrid}
\useinnertheme{rectangles}
\useoutertheme{infolines}

%\usepackage{otf}
%\usepackage[deluxe]{otf}
\usepackage{graphicx}
\usepackage{amsmath}
\usepackage{amssymb}
\usepackage{ascmac}
\usepackage{ulem}
\usepackage{fancybox}
\usepackage{cancel}
\usepackage{plext}
\usepackage{array}
\usepackage[svgnames]{xcolor}
\usepackage{tikz}
\usepackage{accents}
\usepackage{tcolorbox}


\definecolor{titleBLUE}{rgb}{.2,.2,.7}

\def\shatai#1{\makebox[2.25zw][l]{\vphantom{#1}\rotatebox{-48.8}{\scalebox{0.875}[1.143]{\rotatebox{41.2}{\smash{\rlap{#1}}}}}}}

\makeatletter
\def\kenten#1{%
\ifvmode\leavevmode\else\hskip\kanjiskip\fi
%\setbox1=\hbox to \z@{\hspace{0.3zw}{\textrm 、}\hss}
\setbox1=\hbox to \z@{\hspace{0.3zw}${}_{\circ}$\hss}%
\ht1=0.1zw
\@kenten#1\end}
\def\@kenten#1{%
\ifx#1\end \let\next=\relax \else
\raise1.0zw\copy1\nobreak #1\hskip\kanjiskip\relax
\let\next=\@kenten
\fi\next}
\makeatother

\def\vekutoru#1{\mbox{\boldmath $#1$}}

\newcommand{\ubar}[1]{\underaccent{\bar}{#1}}

\newcommand{\bibun}{\mathrm{d}}

\newcommand{\maxprob}{\mathop\mathrm{maximise}\limits}
%\newcommand{\minprob}{\mathop\mathrm{min.}\limits}

\newcommand{\argumentmax}{\mathop\mathrm{arg\,max}\limits}

\def\maru#1{\textcircled{\scriptsize #1}}
\def\shoumon#1{\vspace{1em}\noindent\ovalbox{\textsf{ #1 }}}

\renewcommand{\familydefault}{\sfdefault}
\renewcommand{\kanjifamilydefault}{\gtdefault}
\setbeamerfont{title}{size=\large,series=\bfseries}
\setbeamerfont{frametitle}{size=\large,series=\bfseries}
\setbeamertemplate{frametitle}[default][center]
\usefonttheme{professionalfonts} 

\setbeamertemplate{itemize item}{%
	\footnotesize
	\raise1.0pt
	\hbox{\donotcoloroutermaths$\blacksquare$}
}

\setbeamerfont{itemize/enumerate subbody}{size=\normalsize}
\setbeamertemplate{itemize subitem}{%
	\footnotesize
	\raise1.25pt
	\hbox{\donotcoloroutermaths$\blacktriangleright$}
}

%%%%%%%%%%%%%%%%%%%%%%%%%%%%%%%%%%%%%%%%%%%%%
\newcommand{\KAMOKU}{S. Bowles\quad \shatai{『制度と進化のミクロ経済学』}}
\newcommand{\TAITORU}{2 自生的秩序{\Large : 経済生活の自己組織化}}
\newcommand{\CHOSHA}{下平 勇太}
\newcommand{\CHOSHAen}{Y. Shimodaira}
\newcommand{\SHOZOKU}{大学院安田ゼミ}

\title[Bowles Ch.2]{\TAITORU}
\author[\CHOSHAen]{\CHOSHA}
\institute[]{\SHOZOKU}
\date{\today}

\begin{document}

%%%%%%%%%%%%%%%%%%%%%%%%%%%%%%%%%%%%%%%%%%%%%
\begin{frame}[t]
\vspace{20mm}

\colorbox{titleBLUE}{\vbox {\hsize130mm \quad {\textcolor{white}{
\KAMOKU
}} \vfil}}
{\vbox {\hsize130mm \hspace{20mm} \textbf{\Huge
\vspace{1mm}\\
\hspace*{\stretch{1}}\ \TAITORU
\hspace{\stretch{1}}} \vspace{0mm}}}
\colorbox{titleBLUE}{\vbox {\hsize130mm \hspace*{\stretch{1}}{\textcolor{white}{
\CHOSHA${}^{\dagger}$
}} \quad \vfil}}
\vspace{3mm}

\renewcommand{\thefootnote}{$\dagger$}
\footnotetext[1]{\SHOZOKU.}
\renewcommand{\thefootnote}{*\arabic{footnote}}
\setcounter{footnote}{0}

\tableofcontents

\end{frame}

%%%%%%%%%%%%%%%%%%%%%%%%%%%%%%%%%%%%%%%%%%%%%%
%\newcommand{\secI}{進化社会科学}
%\section{\secI}
%\begin{frame}[t]{\secI}
%
%
%
%\end{frame}

%%%%%%%%%%%%%%%%%%%%%%%%%%%%%%%%%%%%%%%%%%%%%
\newcommand{\secII}{居住分離: 1つの進化過程}
\section{\secII}
\begin{frame}[t]{\secII}
	
\begin{itemize}
	\item 2種類の``人種''(グリーン,ブルー)が存在.
	\item 選好を市内に住むグリーンの割合$f\ \in[0, 1]$上で定義.
	\item 選好を住宅に対する評価(価格) $p_g, p_b \in \mathbb{R}_+$で表現する.
	\item $f^{\#} \succ_{i} f$ ならば,$p_i(f^{\#}) > p_i(f)\ \big( i \in \{g, b\} \big).$
	\item $p_g, p_b$は具体的に,
		\begin{align}
			p_g(f) &= \frac{1}{2}(f-\delta_g) -\frac{1}{2}(f-\delta_g)^2 +p,
			\tag{2.1.g} \\
			p_b(f) &= \frac{1}{2}(f+\delta_b) -\frac{1}{2}(f+\delta_b)^2 +p.
			\tag{2.1.b}
		\end{align}
	\item ただし,$\delta_i \in(0, 1/2)$は差別的好みの度合い.
	\item $p\ (=\text{const.})$は$f$に依存しない,住宅固有の価値.
	\item グリーンにとって,半々より$\delta_g$だけグリーンが多い状態が最適.\\
		ブルーにとって,半々より$\delta_b$だけブルーが多い状態が最適.
\end{itemize}


	
	\big[scale=.8]{fig1}
	\textbf{$f$ v.s. $p_g, p_b$.ただし$\delta_g \neq \delta_b$.$f^*=\frac{1}{2}+\frac{\delta_g - \delta_b}{2}$.}
	\textbf{:fig1}


\end{frame}


%%%%%%%%%%%%%%%%%%%%%%%%%%%%%%%%%%%%%%%%%%%%%
\begin{frame}[t]{\secII}

\begin{itemize}
	\item (人種を問わず)市内の住民のうち$\alpha \in (0, 1)$が市外の住民に住宅を売却するとする.
	\item 市内の住民が潜在的売り手であり,かつブルーである確率は$\alpha (1-f)$.
	\item 市外の人種構成が市内と同じであると仮定.
	\item つまり,市外から来る潜在的買い手がグリーンである確率は$f$.
	\item 潜在的買い手がグリーンであり,かつ,そのグリーンがランダムに訪ねた市内の住民が潜在的売り手であり,かつブルーである確率は
		\begin{equation*}
			\alpha f (1-f).
		\end{equation*}
	\item ある$f$において $p_g(f) > p_b(f)$ のとき,$[\, p_b(f), p_g(f)\, ]$のある価格で売買が成立しうる.
	\item 潜在的買い手のグリーンと潜在的売り手のブルーとの間で売買が成立する確率は,
		\begin{equation*}
			\left\{
			\begin{array}{ll}
				\beta \left[ p_g(f) - p_b(f) \right] & \text{if}\quad p_g(f) > p_b(f), \\
				0 & \text{if}\quad p_g(f) \leq p_b(f).
			\end{array}
			\right.
		\end{equation*}
	\item グリーンとブルーの間で売買が成立すれば$f$が変化.
		変化量は,
		\begin{align}
			\Delta f &=
			\left\{
			\begin{array}{ll}
				\alpha f(1-f)\cdot \beta \left[ p_g(f) - p_b(f) \right]
				& \text{if}\ \text{グリーンがブルーから買う},\\
				-\alpha f(1-f)\cdot \beta \left[ p_b(f) - p_g(f) \right]
				& \text{if}\ \text{ブルーがグリーンから買う},
			\end{array}
			\right. \notag \\
			\notag \\
			&=
			\alpha f(1-f) \rho_g \beta \left[ p_g(f) - p_b(f) \right]
			-\alpha f(1-f) \rho_b \beta \left[ p_b(f) - p_g(f) \right] \tag{2.2} \\
			&=
			\alpha f(1-f) \beta \left[ p_g(f) - p_b(f) \right]. \tag{2.3} \label{eq:2.3}
%			 \\
%			&=
%			\alpha \beta f \left[p_g(f)-\uline{p}\ \right]
%			\quad \big(\uline{p}\equiv fp_g(f)+(1-f)p_b(f)\big). \tag{2.3${}'$}
		\end{align}
		ただし,$\rho_i = 1\ (\text{if}\ p_i>p_{-i})$,$\rho_i =0\ (\text{if}\ p_i\leq p_{-i})$,$\rho_g+\rho_b=1$.
	\item 式(\ref{eq:2.3})がレプリケータ・ダイナミクス方程式.
\end{itemize}

\end{frame}

%%%%%%%%%%%%%%%%%%%%%%%%%%%%%%%%%%%%%%%%%%%%%
\begin{frame}[t]{\secII}
	
\begin{itemize}
	\item $\Delta f$は時間に対する変化量.
		式(\ref{eq:2.3})の右辺は$f$の関数.
		すなわち,式(\ref{eq:2.3})は$f$に関する微分方程式になっている.
	\item 定常状態とは,$f$が時間によって変化しない状態.
		すなわち$\Delta f =0$.
	\item 相図を描いて定常状態の安定性を議論. 
		
			
			\big[scale=.5]{fig2}
			\big[scale=.5]{fig3}
			\textbf{:fig2fig3}
		
		左$\big((\dot{x}(t))'<0\big)$は不安定.
		右$\big((\dot{x}(t))'>0\big)$は安定.
	\item 定常状態$f=f^*$が$\left.\displaystyle\frac{\bibun \Delta f}{\bibun f}\right|_{f=f^*} >0$のとき,その定常状態は不安定.\\
	定常状態$f=f^*$が$\left.\displaystyle\frac{\bibun \Delta f}{\bibun f}\right|_{f=f^*} <0$のとき,その定常状態は安定.
\end{itemize}
	
\end{frame}


%%%%%%%%%%%%%%%%%%%%%%%%%%%%%%%%%%%%%%%%%%%%%
\begin{frame}[t]{\secII}

\begin{itemize}
	\item 簡単のため$\delta_g=\delta_b=\delta$.
	\item 式(\ref{eq:2.3})は
		\begin{equation*}
			\Delta f = \alpha\beta\delta\ f(1-f)(2f-1).
		\end{equation*}
	\item 定常状態$(\Delta f =0)$は,$f=0,\ 1/2,\ 1$の3つ.
	\item 相図を描いて定常状態の安定性を確かめる.\\
		$f=1/2$は不安定.
		$f=0,\ 1$は安定.
		
			
			\big[scale=.8]{fig4}
			\textbf{相図.}
			\textbf{:fig4}
		
	\item $f=1/2$とは,グリーンとブルーが半々で混住する均衡.
		しかし,僅かにバランスが崩れれば,たちまち$f=1/2$は維持できなくなる.
	\item $f=1$とは,市内にグリーンしかいなくなる均衡.
		偶然,ブルーが入ってきても,すぐにグリーンに入れ替わる.
\end{itemize}
	
	
\end{frame}


%%%%%%%%%%%%%%%%%%%%%%%%%%%%%%%%%%%%%%%%%%%%%
\begin{frame}[t]{\secII}
	
\begin{itemize}
	\item 例えば図の場合,$p_g(1/2)>p_g(1)$,$p_b(1/2)>p_b(1)$より,$f=1$は$f=1/2$に対してPareto劣位.
		
			
			\big[scale=.8]{fig5}
			\textbf{:fig5}
		
	\item $f=1$が$f=1/2$に対してPareto劣位である条件は,
		\begin{equation*}
			p_g\left(\frac{1}{2}\right) > p_g(1) \quad \Longrightarrow\quad (0<)\ \delta < \frac{1}{4}.
		\end{equation*}
\end{itemize}
	
\end{frame}

%%%%%%%%%%%%%%%%%%%%%%%%%%%%%%%%%%%%%%%%%%%%%
\newcommand{\secIII}{行動進化のモデル化}
\section{\secIII}
\begin{frame}[t]{\secIII}

\begin{itemize}
	\item 個体群の個人は,$x$型か$y$型のいずれか一方の特性を持つ.
	\item 相手が$j$をプレイするとき,自分が$i$をプレイするときの利得$\pi(i, j)$.
		ただし,$i, j \in \{x, y\}$.
	\item 個体群の中で$x$型が占める割合は$p \ \in [0, 1]$.
	\item $x$型,$y$型それぞれをプレイするときの期待利得は
		\begin{align}
			b_x(p) &= p\pi(x, x) + (1-p) \pi(x, y), \tag{2.4.x} \\
			b_y(p) &= p\pi(y, x) + (1-p) \pi(y, y). \tag{2.4.y}
		\end{align}
	\item 個体群のうち,$\omega \ \in [0, 1]$が文化モデルに接触して自らの特性を更新する\kenten{可能性がある}.	
	\item ある$p$において$b_y(p)>b_x(p)$のとき,$x$型は自らの特性を$y$型に変えることにやぶさかではない.
	\item ある$p$において$b_y(p)\leq b_x(p)$のとき,$x$型は自らの特性を$y$型に絶対変えない.
	\item $x$型が自らを$y$型に更新する確率は,
		\begin{equation*}
			\left\{
			\begin{array}{ll}
				\beta \left[ b_y(p) - b_x(p) \right] & \text{if}\quad b_y(p) > b_x(p), \\
				0 & \text{if}\quad b_y(p) \leq b_x(p).
			\end{array}
			\right.
		\end{equation*}
	\item $\rho$を定義.
		\begin{equation*}
			\rho = \left\{
			\begin{array}{ll}
				1 & \text{if}\quad b_y(p) > b_x(p), \\
				0 & \text{if}\quad b_y(p) \leq b_x(p).
			\end{array}
		\right.
		\end{equation*}
	\item 隣接2期間での$p$の変化量$\Delta p$は,
		\begin{align}
			\Delta p 
			&=
			\left\{
			\begin{array}{llll}
				-
				&\underbrace{\omega p}_{\text{特性を変えうる$x$が}}
				&\cdot \underbrace{(1-p)}_{\text{文化モデル$y$と接触し}}
				&\cdot \underbrace{\beta \left[ b_y(p) - b_x(p) \right]}_{\text{実際に特性を変える.}},\\
				&\underbrace{\omega (1-p)}_{\text{特性を変えうる$y$が}}
				&\cdot \underbrace{p}_{\text{文化モデル$x$と接触し}}
				&\cdot \underbrace{\beta \left[ b_x(p) - b_y(p) \right]}_{\text{実際に特性を変える.}},
			\end{array}
			\right. \notag \\
			\notag \\
			&=
			-\omega p(1-p) \beta 
			\left[ b_y(p) - b_x(p) \right] \rho
			+\omega p(1-p) \beta 
			\left[ b_x(p) - b_y(p) \right] (1-\rho),
			\tag{2.5} \\
			&=
			p(1-p) \ \omega \beta
			\left[ b_x(p) - b_y(p) \right].
			\tag{2.6}
		\end{align}
\end{itemize}


\end{frame}


%%%%%%%%%%%%%%%%%%%%%%%%%%%%%%%%%%%%%%%%%%%%%
\begin{frame}[t]{\secIII}

\begin{itemize}
	\item レプリケータ・ダイナミクスの一般形
		\begin{equation}
			\Delta p = p(1-p) \ \omega \beta
			\left[ b_x(p) - b_y(p) \right].
			\tag{2.6}
			\label{eq:2.6}
		\end{equation}
	\item 個体群から無作為抽出した個人が$x$型であるか否か,という確率変数はBernouli分布に従う.
	\item $p(1-p)$はBernouli分布の分散.
	\item $p$が極端な値の時,分散は小さく,個体群の均質性が高い.
	\item $\therefore$ 個体群の均質性が高い時,進化の過程はゆっくり($\Delta p$が小).
	\vspace{5mm}
	\item $\omega \beta
	\left[ b_x(p) - b_y(p) \right]$は利得に対する$p$の効果を表す.
	\item $\therefore$ $b_x(p)$と$b_y(p)$の差が大きければ,進化の過程はスピーディー.
	\vspace{5mm}
	\item 個体群全体における平均利得$\uline{b}(p)\equiv pb_x(p)+(1-p)b_y(p)$を定義.
	\item これを用いて式(\ref{eq:2.6})を書き直すと,
		\begin{equation}
			\Delta p = p\omega\beta[b_x(p)-\uline{b}(p)].
			\tag{2.6$'$}
		\end{equation}
	\item これが離散時間型レプリケータ・ダイナミクスの一般形.
\end{itemize}

	
\end{frame}

%%%%%%%%%%%%%%%%%%%%%%%%%%%%%%%%%%%%%%%%%%%%%
\begin{frame}[t]{\secIII}

\begin{itemize}
	\item レプリケータ・ダイナミクスの一般形
		\begin{equation}
			\Delta p = p(1-p) \ \omega \beta
			\left[ b_x(p) - b_y(p) \right],
			\tag{2.6}
		\end{equation}
		where
		\begin{align}
			b_x(p) &= p\pi(x, x) + (1-p) \pi(x, y), \tag{2.4.x} \\
			b_y(p) &= p\pi(y, x) + (1-p) \pi(y, y). \tag{2.4.y}
		\end{align}
	\item 式(\ref{eq:2.6})の右辺は$p$の関数.
		これを$\gamma(p)$と書くことにしよう.
	\item $\pi(\cdot, \cdot) =\text{const.}$なら,$\gamma(p)$は$p$に関して3次関数.
	\vspace{5mm}
	\item $\Delta p = \gamma(p^*) =0$を満たす状態$p^*$を定常状態と呼ぶ.
	\item 式(\ref{eq:2.6})より,定常状態となる$p$は3つの場合がある:
		\begin{equation*}
			p=0,\quad
			p=1,\quad
			p=p^{\#}\ \text{s.t.} \ b_x(p^{\#}) = b_y(p^{\#}).
		\end{equation*}
	\vspace{0mm}
	\item $p(1-p) >0$,$\omega\beta >0$より,$\Delta p$の符号は$b_x-b_y$の符号に等しい.
	\item $\therefore$ 更新は利得単調的.i.e., $x \succ y\ \Longleftrightarrow \ \Delta p > 0$.
	\item 平均以上の利得を伴う行動(特性)が他の人々に模倣され,それが個体群を占める割合が増大する.
\end{itemize}



\end{frame}

%%%%%%%%%%%%%%%%%%%%%%%%%%%%%%%%%%%%%%%%%%%%%
\begin{frame}[t]{\secIII}
	
\begin{itemize}
	\item 定常状態$p^*$において,摂動$\epsilon$を考える.
	\item $p^*$が漸近安定であるとは,
		\begin{equation*}
			\gamma(p^*+\epsilon) \lessgtr 0 
			\quad \big(\epsilon \gtrless 0\big).
		\end{equation*}
		$\epsilon\gtrless0$ずれたら,$\Delta p = \gamma(p^*+\epsilon)\lessgtr0$戻る.
	\item すなわち,
		\begin{equation*}
			\underbrace{0 > \frac{\gamma(p^*+\epsilon)}{\epsilon}}%
			_{\substack{\text{``$\lessgtr$''どちらの場合も}\\ \text{この不等式で済む.}}}
			 \ =
			\frac{\gamma(p^*+\epsilon) - \gamma(p^*)}{\epsilon}\ 
			\xrightarrow[\epsilon\rightarrow 0]{}
			\frac{\bibun \gamma(p^*)}{\bibun p}
			= \left.\frac{\bibun \Delta p}{\bibun p}\right|_{p=p^*}.
		\end{equation*}
%	\vspace{5mm}
%	\item $p^*$がLyapunov安定であるとは,$p^*$の近くから出発するどんな経路も,$p^*$の近くに留まり続けている状態であること.
%	\item (ちょっと難しく言うと,)\\
%		$\forall \epsilon >0$に対して $\exists \delta >0$,s.t.
%		\begin{equation*}
%			\| p_0 - p^* \| < \delta
%			\ \Longrightarrow \ 
%			(\forall t)\ \| p_t - p^* \| < \epsilon
%		\end{equation*}
%		ならば,$p^*$はLyapunov安定.
%	\item 漸近安定 $\Longrightarrow$ Lyapunov安定.
%	\vspace{5mm}
%	\item 定常状態$p=p^*$が$(0, 1)$の内点で,かつ,不安定なとき,
%		\begin{equation*}
%			\left\{
%			\begin{array}{ll}
%				\Delta p <0 \ \Rightarrow \ (0<)\ p<p^* \ :
%				&\quad p=0\text{の吸引域}, \\
%				\Delta p >0 \ \Rightarrow \ p^*<p\ (<1) \ :
%				&\quad p=1\text{の吸引域}.
%			\end{array}
%			\right.
%		\end{equation*}
%\end{itemize}
%	
%	
%\end{frame}
%
%
%%%%%%%%%%%%%%%%%%%%%%%%%%%%%%%%%%%%%%%%%%%%%%
%\begin{frame}[t]{\secIII}
%
%\begin{itemize}
	\vspace{5mm}
	\item \textcolor{magenta}{内点$p^{\#} \ \in(0, 1)$が}漸近安定であるための必要条件は,
		\begin{equation*}
			\frac{\bibun b_x(p)}{\bibun p} < \frac{\bibun b_y(p)}{\bibun p},
		\end{equation*}
		\begin{equation}
			\therefore\quad
			\pi(y,x) -\pi(y,y) -\pi(x,x) +\pi(x,y)>0.
			\tag{2.8}
			\label{eq:2.8}
		\end{equation}
	\item 期待利得$b_i(p)$の傾きを比較.
		$p$の上昇(i.e., $x$の増加)を打ち消すために,$b_y$の傾きの方が大きくないとダメ.
	\item ($\because$)\quad $p^{\#}$が漸近安定であるとき,以下が成立.
		\begin{equation*}
			\left\{
			\begin{array}{ll}
				b_x(p) > b_y(p) 
				& \text{if}\quad (0<)\, p < p^{\#}, \\
				b_x(p) < b_y(p)
				& \text{if}\quad p^{\#} < p \,(<1).
			\end{array}
			\right.
		\end{equation*}
		このためには式(\ref{eq:2.8})が必要(グラフで理解).
		\begin{align*}
			0&> \left.\frac{1}{\omega\beta}
			\frac{\bibun \Delta p}{\bibun p}\right|_{p=p^{\#}}
			= (1-2p)
			\underbrace{[b_x(p^{\#})-b_y(p^{\#})]}_{=0} \\
			&\hspace{40mm}
			+\underbrace{p(1-p)}_{>0}
			\frac{\bibun}{\bibun p}[b_x(p^{\#})-b_y(p^{\#})] \\
			\therefore\ 0&>
			\frac{\bibun}{\bibun p}[b_x(p^{\#})-b_y(p^{\#})].
		\end{align*}
\end{itemize}


\end{frame}

%%%%%%%%%%%%%%%%%%%%%%%%%%%%%%%%%%%%%%%%%%%%%
\newcommand{\secIV}{進化的な安定性と社会的結果}
\section{\secIV}
\begin{frame}[t]{\secIV}

\begin{itemize}
	\item 以下の利得表で表されるようなタカ・ハトゲームを考える.
		\begin{equation*}
			\begin{array}{lcc}
				\hline\hline\\[-12pt]
				& H & D \\ \hline \\[-10pt]
				\text{タカ} H & -10 & 10 \\[2mm]
				\text{ハト} D & 0 & 5 \\[2mm]
				\hline\hline
			\end{array}
		\end{equation*}
	\item 純粋戦略で対称Nash均衡は存在しない.
	\item 混合戦略で対称Nash均衡は$\ldots$
	\item タカの割合を$q$,ハトの割合を$1-q$とする.$q \in[0, 1]$.
	\item タカ,ハト,それぞれの戦略を採るときの期待利得$u_H$,$u_D$は,
		\begin{align*}
			u_H(q) &= q \pi(H, H) + (1-q) \pi(H, D) \\
			&= -10q+10(1-q) = -20q+10, \\
			u_D(q) &= q \pi(D, H) + (1-q) \pi(D, D) \\
			&= 0q+5(1-q) = -5q+5.
		\end{align*}
	\item タカ,ハト両戦略が無差別になるのは,
		\begin{equation*}
			\begin{array}{rrcl}
				& u_H(q) & = & u_D(q) \\
				\Longrightarrow \quad 
				& -20q+10 & = & -5q+5 \\
				\Longrightarrow \quad 
				& q & = & \dfrac{1}{3}.
			\end{array}
		\end{equation*}
	\item したがって,対称Nash均衡$s^*=(s^*(H), s^*(D))$は,
		\begin{equation*}
			(s^*(H), s^*(D)) = \left(\frac{1}{3}, \frac{2}{3} \right).
		\end{equation*}
\end{itemize}


\end{frame}

%%%%%%%%%%%%%%%%%%%%%%%%%%%%%%%%%%%%%%%%%%%%%
\begin{frame}[t]{\secIV}

\begin{itemize}
	\item $u_H$,$u_D$をグラフに描く.
		
			
			\big[scale=.8]{fig6}
			\textbf{:fig6}
		
	\item $q<s^*(H)\ (=1/3)$の領域では$u_H>u_D$.
	\item タカの期待利得の方が大きいからタカが増える.
		i.e., $q$は大きくなる.
	\item $q>s^*(H)\ (=1/3)$の領域では$u_H<u_D$.
	\item ハトの期待利得の方が大きいからハトが増える.
		i.e., $q$は小さくなる.
	\item ちょうど$q=s^*(H)\ (=1/3)$では,一方の割合を大きくしたときに,他方の戦略の期待利得が大きくなって,元の割合に戻すチカラが働く.
	\item これを進化的に安定という.
\end{itemize}


	
\end{frame}


%%%%%%%%%%%%%%%%%%%%%%%%%%%%%%%%%%%%%%%%%%%%%
\begin{frame}[t]{\secIV}
	
\begin{itemize}
	\item 以下の利得表で表されるような協調ゲームを考える.
		\begin{equation*}
			\begin{array}{lcc}
				\hline\hline\\[-12pt]
				& M & L \\ \hline \\[-10pt]
				\text{Mac}\ M & 6 & 0 \\[2mm]
				\text{Linux}\ L & 0 & 4 \\[2mm]
				\hline\hline
			\end{array}
		\end{equation*}
	\item 純粋戦略で対称Nash均衡は,全員が$M$を選ぶか,または全員が$L$を選ぶこと.
	\item 混合戦略で対称Nash均衡は$\ldots$
	\item Macの割合を$q$,Linuxの割合を$1-q$とする.
		$q \in[0, 1]$.
	\item Mac,Linux,それぞれの戦略を採るときの期待利得$u_M$,$u_L$は,
		\begin{align*}
			u_M(q) &= q \pi(M, M) + (1-q) \pi(M, L) \\
			&= 6q+0(1-q) = 6q, \\
			u_L(q) &= q \pi(L, M) + (1-q) \pi(L, L) \\
			&= 0q+4(1-q) = -4q+4.
		\end{align*}
	\item Mac,Linux両戦略が無差別になるのは,
		\begin{equation*}
			\begin{array}{rrcl}
				& u_M(q) & = & u_L(q) \\
				\Longrightarrow \quad 
				& 6q & = & -4q+4 \\
				\Longrightarrow \quad 
				& q & = & \dfrac{2}{5}.
			\end{array}
		\end{equation*}
	\item したがって,(混合戦略)対称Nash均衡$s^*=(s^*(M), s^*(L))$は,
		\begin{equation*}
			(s^*(M), s^*(L)) = \left(\frac{2}{5}, \frac{3}{5} \right).
		\end{equation*}
\end{itemize}
	
	
\end{frame}

%%%%%%%%%%%%%%%%%%%%%%%%%%%%%%%%%%%%%%%%%%%%%
\begin{frame}[t]{\secIV}
	
	\begin{itemize}
		\item $u_M$,$u_L$をグラフに描く.
		
			
			\big[scale=.8]{fig7}
			\textbf{:fig7}
		
		\item $q<s^*(M)\ (=2/5)$の領域では$u_M<u_L$.
		\item Linuxの期待利得の方が大きいからLinuxが増える.i.e., $q$は小さくなる.
		\item $q>s^*(M)\ (=2/5)$の領域では$u_M>u_L$.
		\item Macの期待利得の方が大きいからMacが増える.i.e., $q$は大きくなる.
		\item ちょうど$q=s^*(M)\ (=2/5)$\kenten{以外}では,一方の割合を大きくしたときに,その戦略の期待利得が大きくなって,元の割合から離れるチカラが働く.
		\item これは進化的に安定ではない.
		\item 一方,$s^*=(0, 1), (1, 0)$は進化的に安定である.
	\end{itemize}
	
	
	
\end{frame}


%%%%%%%%%%%%%%%%%%%%%%%%%%%%%%%%%%%%%%%%%%%%%
\begin{frame}[t]{\secIV}

\begin{itemize}
	\item 個体群の個体はタカ($H$)かハト($D$)のいずれかであるとする.$\Sigma=\{H, D\}$
	\item タカ1羽とハト1羽が出会ったとき,それぞれ,$\pi(H, D)$羽,$\pi(D, H)$羽の子孫を(無性生殖によって)残すとする.
	\item たとえば$\pi(H, D)=10$,$\pi(D, H)=0$のとき,$H:D=1:1$から$H:D=11:1$に変化する.
	\item $\pi(i, j)$を適応度(関数)と呼ぶ.ただし$i\in\Sigma$は自分が持っている戦略,$j\in\Sigma$は相手が持っている戦略.
	\item 個体群の中の$H$の割合が$p$,$D$の割合が$1-p$のとき,自分が$i\in\Sigma$を選んだときの期待適応度は
		\begin{equation*}
			p\pi(i, H) + (1-p)\pi(i, D).
		\end{equation*}
\end{itemize}

\begin{definition}[進化的に安定な戦略: ESS]
	\quad
	戦略$s^*\in\Sigma$が進化的に安定な戦略であるとは,任意の他の戦略$s \in\Sigma\setminus\{s^*\}$について,ある$\tilde{p}>0$が存在し,任意の$\epsilon \in(0, \tilde{p})$について
		\begin{equation}
			(1-\epsilon)\pi(s^*, s^*) + \epsilon\pi(s^*, s)
			>
			(1-\epsilon)\pi(s, s^*) + \epsilon\pi(s, s)
			\label{eq:ESS}
		\end{equation}
	が成立すること.
\end{definition}

\begin{itemize}
	\item この定義を以下のように読み下す.
	\item 個体群の個体がすべて$s^*$を戦略に持っている中,わずかに($\epsilon$だけ) 変異体$s$が侵入したとする.
	\item このときの$s^*$の期待適応度が左辺.
	\item このときの$s$の期待適応度が右辺.
	\item どんな変異体$s$も,十分に小さい割合$\epsilon$で侵入してくる場合,ESS $s^*$の期待適応度の方が厳密に大きくなることを要求.
	\vspace{3mm}
	\item 利得最大化 $=$ 合理的選択.
	\item 適応度最大化 $=$ 自然淘汰.
	\item Nash均衡を,合理性を前提としない均衡として再解釈.
\end{itemize}
	
\end{frame}


%%%%%%%%%%%%%%%%%%%%%%%%%%%%%%%%%%%%%%%%%%%%%
\begin{frame}[t]{\secIV}
	
\begin{itemize}
	\item 式(\ref{eq:ESS})を整理してみると,
		\begin{equation*}
			(1-\epsilon)\Big[\pi(s^*, s^*) - \pi(s, s^*)\Big]
			+\epsilon\Big[\pi(s^*, s) - \pi(s, s)\Big]
			>0.
		\end{equation*}
	\item $1-\epsilon\gg\epsilon$に注意すると,	$s^*$がESSである条件は
		\begin{block}{ESSの条件}
			\begin{equation}
				\pi(s^*, s^*) > \pi(s, s^*)
				\label{eq2}
			\end{equation}
			または
			\begin{equation}
				\pi(s^*, s^*) = \pi(s, s^*)
				\quad\text{かつ}\quad
				\pi(s^*, s) > \pi(s, s)
				\label{eq3}
			\end{equation}
			が成り立つこと
		\end{block}
		と書きなおすことができる.
	\item 前半の式より,$s^*$は対称Nash均衡になっていることが分かる.
	\item ESS $\Longrightarrow$ 対称Nash均衡.
%	\item ESS $\centernot\Longleftarrow$ 対称Nash均衡.
\end{itemize}
	
\end{frame}



%%%%%%%%%%%%%%%%%%%%%%%%%%%%%%%%%%%%%%%%%%%%%
\begin{frame}[t]{\secIV}

\begin{itemize}
	\item 個体群の個体は,$x$型か$y$型のいずれか一方の特性を持つ.
	\item 個体群を占める$x$型の割合を$p$.
%	\item 侵入障壁$\tilde{p}\ \in(0, 1)$は$p=0$の吸引域と$p=1$の吸引域の境界.
	\item \kenten{全員が}$y$型である個体群に,少数$\epsilon$の$x$型が侵入する状況を考える.
	\item R.D.方程式$\Delta p =\gamma(p)$を$p=\epsilon$で評価する.
		\begin{equation*}
			\Delta p\Big|_{p=\epsilon} = \epsilon(1-\epsilon) \ \omega \beta
			\left[ b_x(\epsilon) - b_y(\epsilon) \right].
		\end{equation*}
	\item $\Delta p$の符号は$b_x-b_y$の符号と等しい.
		\begin{align*}
			b_x(\epsilon)-b_y(\epsilon) &=
			[\epsilon\pi(x, x)+(1-\epsilon)\pi(x, y)]
			-[\epsilon\pi(y, x)+(1-\epsilon)\pi(y, y)] \\
			&=
			(1-\epsilon)[\pi(x, y) - \pi(y, y)]
			+\epsilon[\pi(x, x) - \pi(y, x)].
		\end{align*}
	\item $\epsilon$が十分小さいとき,$1-\epsilon \gg \epsilon$ゆえ
		\begin{equation*}
			b_x(\epsilon)-b_y(\epsilon) \ 
			\left\{
			\begin{array}{ll}
				<0 &
				\text{if}\quad
				\textcolor{red}{
				\pi(x, y) < \pi(y, y)}, \\
				<0 &
				\text{if}\quad
				\textcolor{red}{
				\pi(x, y) = \pi(y, y)
				\ \text{かつ}\ 
				\pi(x, x) < \pi(y, x)}, \\
				=0 &
				\text{if}\quad
				\pi(x, y) = \pi(y, y)
				\ \text{かつ}\ 
				\pi(x, x) = \pi(y, x), \\
				>0 &
				\text{if}\quad
				\pi(x, y) = \pi(y, y)
				\ \text{かつ}\ 
				\pi(x, x) > \pi(y, x), \\
				>0 &
				\text{if}\quad
				\pi(x, y) > \pi(y, y).
			\end{array}
			\right.
		\end{equation*}
	\item 任意の小さい$\epsilon>0$に対して$\Delta p |_{p=\epsilon}<0$ならば,``全員が$y$型''は漸近安定的な均衡.
	\vspace{5mm}
	\item ところで,侵入者$x\,(=s)$に対してマジョリティ$y\,(=s^*)$がESSである条件は,式(\ref{eq2})または式(\ref{eq3})が成立すること.
	\item すなわち,任意の小さい$\epsilon>0$に対して,
		\begin{equation}
			\textcolor{red}{
			\pi(y, y) > \pi(x, y)
			}
			\tag{2.9.1}
		\end{equation}
	または,
		\begin{equation}
			\textcolor{red}{
			\pi(y, y) = \pi(x, y)
			\quad \text{かつ}\quad
			\pi(y, x) > \pi (x, x)
			}
			\tag{2.9.2}
		\end{equation}
	が成り立つこと.
	\vspace{5mm}
	\item 結局,ESSはレプリケータ・ダイナミクスにおける漸近安定的な均衡.
%	\item 各$j \ \in\{x, y\}$に対して$\pi(y, j) \geq \pi(x, j)$が成立しているから,$y$は最適反応になっている.
%	\item $x$に侵入されてもそれ以上増殖しない:\ $y$は中立的安定状態(NSS).
%	\item ESS $\Longrightarrow$ NSS.
%	\item NSS $\Longrightarrow$ Nash均衡.
\end{itemize}
	

\end{frame}


%%%%%%%%%%%%%%%%%%%%%%%%%%%%%%%%%%%%%%%%%%%%%
\begin{frame}[t]{\secIV}
	
\begin{itemize}
	\item タカ・ハトゲーム.利得表は以下.
		\begin{equation*}
			\begin{array}{lll}
				\hline\hline\\[-12pt]
				& \quad h & \quad d \\ \hline \\[-10pt]
				\text{タカ} h & (A=)\ \dfrac{1}{2}(v-c) & (B=)\ v \\[2mm]
				\text{ハト} d & (C=)\ 0 & (D=)\ \dfrac{1}{2}v \\[2mm]
				\hline\hline
			\end{array}
		\end{equation*}
	\item タカの割合を$p$とする.
	\item タカ,ハトそれぞれの戦略を選んだときの期待利得$b_h$,$b_d$は
		\begin{align}
			b_h(p) &= pA + (1-p)B, \tag{2.10.h} \\
			b_d(p) &= pC + (1-p)D. \tag{2.10.d}
		\end{align}
	\item 来期,タカ1羽は$b_h+\phi$の,ハト1羽は$b_d+\phi$の子を作る.
	\item 来期のタカの割合$p'$は,
		\begin{equation}
			p' = \frac{p(b_h + \phi)}{p(b_h + \phi) +(1-p)(b_d + \phi)}
			= \frac{p(b_h + \phi)}{pb_h +(1-p)b_d + \phi}. 
			\tag{2.11}
		\end{equation}
	\item 変化量$\Delta p$は,
		\begin{align}
			\Delta p &= p' - p \notag \\
			&=
			\frac{p(b_h + \phi) - p[p(b_h + \phi) +(1-p)(b_d + \phi)]}{pb_h +(1-p)b_d + \phi}
			\tag{2.12} \\
			&=
			{\uline{b}}^{-1}
			\ p(1-p)(b_h-b_d) \notag \\
			&=
			{\uline{b}}^{-1}
			\ \frac{1}{2}p(1-p)(v-pc).
			\tag{2.12${}'$} \\
			&
			\hspace{15mm}
			\big(\because\ b_h-b_d=p(A-C)+(1-p)(B-D)=(v-pc)/2.\big)
			\notag
		\end{align}
		ただし$\uline{b}\equiv pb_h +(1-p)b_d + \phi$.
\end{itemize}

	
\end{frame}


%%%%%%%%%%%%%%%%%%%%%%%%%%%%%%%%%%%%%%%%%%%%%
\begin{frame}[t]{\secIV}
	
\begin{itemize}
	\item $\uline{b}\Delta p = \dfrac{1}{2}p(1-p)(v-pc)$
		はR.D.方程式になっている.
	\item $\Delta p = 0$を満たす$p$は,
		\begin{equation}
			p=0,\ 1,\ p^*=\frac{v}{c}.
			\tag{2.13}
		\end{equation}
%	\item $v$は獲物を得る利得.$c$はタカ同士戦うコスト.
	\item $b_h=b_d$を満たす内点$p^*$は安定的.なんとなればすなわち
		\begin{equation}
			\frac{\bibun}{\bibun p}[b_h(p)-b_d(p)]
			=-\frac{1}{2}c <0.
			\tag{2.14}
		\end{equation}
	\item $b_h(0)=v>v/2=b_d(0)$ゆえ,$p=0$はタカに逸脱のインセンティヴがあり,Nash均衡ではない.
	\item $b_h(1)=(v-c)/2<0=b_d(1)$ゆえ,$p=1$はハトに逸脱のインセンティヴがあり,Nash均衡ではない.
	\vspace{5mm}
	\item 一般に以下のようにまとめられる.
		\begin{equation*}
			\begin{array}{lll}
				\hline\hline\\[-12pt]
				& y \text{はESS} & y \text{はESSでない} \\ \hline \\[-10pt]
				x \text{はESS} & p^* \in(0, 1) \text{は不安定} & p=1 \text{は安定} \\[2mm]
				x \text{はESSでない} & p=0 \text{は安定} & p^* \in(0, 1) \text{は安定} \\[2mm]
				\hline\hline
			\end{array}
		\end{equation*}
	\item $x$, $y$いずれもESSのとき,どっちが実際に選ばれる? 均衡選択の問題.
\end{itemize}


\end{frame}

%%%%%%%%%%%%%%%%%%%%%%%%%%%%%%%%%%%%%%%%%%%%%%
%\begin{frame}[t]{\secIV}
%	
%	
%\end{frame}
%
%
%
%%%%%%%%%%%%%%%%%%%%%%%%%%%%%%%%%%%%%%%%%%%%%%
%\newcommand{\secV}{所有権の進化}
%\section{\secV}
%\begin{frame}[t]{\secV}
%	
%\begin{itemize}
%	\item 
%\end{itemize}
%	
%\end{frame}
%
%
%%%%%%%%%%%%%%%%%%%%%%%%%%%%%%%%%%%%%%%%%%%%%%
%\begin{frame}[t]{\secV}
%	
%	
%	
%	
%	
%\end{frame}



\end{document}