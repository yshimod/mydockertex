\documentclass[uplatex,14pt,dvipdfmx,xcolor=svgnames]{beamer}
\usepackage[sourcehan]{pxchfon}
\usepackage{bxdpx-beamer}
\usepackage{minijs}

%\setbeamertemplate{navigation symbols}{}

\usepackage{atbegshi}
\AtBeginDvi{\special{pdf:tounicode EUC-UCS2}}

\usetheme{Madrid}
\useinnertheme{rectangles}
\useoutertheme{infolines}

%\usepackage{otf}
%\usepackage[deluxe]{otf}
\usepackage{graphicx}
\usepackage{amsmath}
\usepackage{amssymb}
\usepackage{ascmac}
\usepackage{ulem}
\usepackage{fancybox}
\usepackage{cancel}
\usepackage{plext}
\usepackage{array}
\usepackage[svgnames]{xcolor}
\usepackage{tikz}
\usepackage{accents}
\usepackage{tcolorbox}

\makeatletter
\def\kenten#1{%
\ifvmode\leavevmode\else\hskip\kanjiskip\fi
%\setbox1=\hbox to \z@{\hspace{0.3zw}{\textrm 、}\hss}
\setbox1=\hbox to \z@{\hspace{0.32zw}${}_{\circ}$\hss}%
\ht1=0.1zw
\@kenten#1\end}
\def\@kenten#1{%
\ifx#1\end \let\next=\relax \else
\raise1.0zw\copy1\nobreak #1\hskip\kanjiskip\relax
\let\next=\@kenten
\fi\next}
\makeatother

\def\vekutoru#1{\mbox{\boldmath $#1$}}

\newcommand{\ubar}[1]{\underaccent{\bar}{#1}}

\newcommand{\bibun}{\mathrm{d}}

\newcommand{\maxprob}{\mathop\mathrm{maximise}\limits}
%\newcommand{\minprob}{\mathop\mathrm{min.}\limits}

\newcommand{\argumentmax}{\mathop\mathrm{arg\,max}\limits}

\def\maru#1{\textcircled{\scriptsize #1}}
\def\shoumon#1{\vspace{1em}\noindent\ovalbox{\textsf{ #1 }}}

\renewcommand{\familydefault}{\sfdefault}
\renewcommand{\kanjifamilydefault}{\gtdefault}
\setbeamerfont{title}{size=\large,series=\bfseries}
\setbeamerfont{frametitle}{size=\large,series=\bfseries}
\setbeamertemplate{frametitle}[default][center]
\usefonttheme{professionalfonts}  

\setbeamertemplate{itemize item}{%
	\footnotesize
	\raise1.0pt
	\hbox{\donotcoloroutermaths$\blacksquare$}
}

%\setbeamerfont{itemize/enumerate subbody}{size=\normalsize}
\setbeamertemplate{itemize subitem}{%
	\footnotesize
	\raise1.25pt
	\hbox{\donotcoloroutermaths$\blacktriangleright$}
}


%%%%%%%%%%%%%%%%%%%%%%%%%%%%%%%%%%%%%%%%%%%%%

\title[Grossman \& Hart (1986)]{The Costs and Benefits of Ownership: \\ A Theory of Vertical and Lateral Integration}
\subtitle{Grossman \& Hart (1986 JPE)}
\author[Shimodaira, Y.]{Yuta Shimodaira}
\institute[]{}
\date{12 June 2017}

%%%%%%%%%%%%%%%%%%%%%%%%%%%%%%%%%%%%%%%%%%%%%
%%%%%%%%%%%%%%%%%%%%%%%%%%%%%%%%%%%%%%%%%%%%%
\begin{document}

\begin{frame}[c]
	
\maketitle
	
\end{frame}


%%%%%%%%%%%%%%%%%%%%%%%%%%%%%%%%%%%%%%%%%%%%%
\begin{frame}[t]
	
	\tableofcontents
	
\end{frame}


%%%%%%%%%%%%%%%%%%%%%%%%%%%%%%%%%%%%%%%%%%%%%

\section{4.1 (omitted)}

%%%%%%%%%%%%%%%%%%%%%%%%%%%%%%%%%%%%%%%%%%%%%
\newcommand{\secII}{4.2 The Received Auction Theory}
\section{\secII}
\newcommand{\ssecIIa}{4.2.1 Relevance of the Received Theory}
\subsection{\ssecIIa}
\begin{frame}[t]{\ssecIIa}
	
\begin{itemize}
	\item Assumptions of the RET:
	\begin{itemize}
		\item bidders are symmetric;
		\item bidders are risk-neutral;
		\item bidders share common priors;
		\item bidders play non-cooperative Nash eqm.;
		\item the number of bidders are independent;
		\item the types of bidders are independent.
	\end{itemize}
\end{itemize}
	
\end{frame}

%%%%%%%%%%%%%%%%%%%%%%%%%%%%%%%%%%%%%%%%%%%%%
\begin{frame}[t]{\ssecIIa\textcolor[rgb]{0.2,0.2,0.7}{text}}
	
\begin{itemize}
%	\item The theoretical result that if
	\item If information is non-independent, the ascending auctions are more profitable than first-price sealed-bid auctions.
	\item An explanation of this is that bidders' profits derive from their private information, and the auctioneer can profit by reducing that private information.
%	\item The ascending auction is more profitable. 
\end{itemize}
	
\end{frame}

%%%%%%%%%%%%%%%%%%%%%%%%%%%%%%%%%%%%%%%%%%%%%
\begin{frame}[t]{\ssecIIa}
	
\begin{itemize}
	\item Riley \& Li (1997): numerical analysis suggests that the effects of affiliation are often tiny.
	\begin{itemize}
		\item Unless the information is very stlongly affiliated, the revenue difference between ascending and first-price auction is very small.
	\end{itemize}
	\item Perry \& Reny (1999): the result about affiliation does not hold in multi-unit auctions {\small (even in theory)}.
	\item Argues the affiliation effect is important.
\end{itemize}
	
\end{frame}

%%%%%%%%%%%%%%%%%%%%%%%%%%%%%%%%%%%%%%%%%%%%%
\newcommand{\secIII}{4.3 The Elementary Economic Theory That Matters}
\section{\secIII}
\newcommand{\ssecIIIa}{4.3.1 Entry}
\subsection{\ssecIIIa}
\begin{frame}[t]{\ssecIIIa}

\begin{itemize}
	\item The profitability depends on the number of participants.
	\item Different auctions vary in participants attractiveness to entry.
	\vspace{5mm}
	\item Bidders will only undertake if they feel that they have chances of winning.
	\item Participating can be costly exercise.
\end{itemize}

\end{frame}

%%%%%%%%%%%%%%%%%%%%%%%%%%%%%%%%%%%%%%%%%%%%%
\begin{frame}[t]{\ssecIIIa}

\begin{itemize}
	\item In \emph{an ascending auction}, a stronger bidder can always top any bid.
	\item Knowing this, a weaker bidder may not enter.
	\item Then, the stronger bidder is able to win at a low price.
	\vspace{3mm}
	\item In \emph{a first-price seald-bid auction}, a weaker bidder may win.
	\item A stronger bidder could have beaten, but did not.
	\item The stronger bidder may risk trying to win at a lower price.
	\vspace{3mm}
	\item More bidders may enter a first-price sealed-bid auction
\end{itemize}

\end{frame}


%%%%%%%%%%%%%%%%%%%%%%%%%%%%%%%%%%%%%%%%%%%%%
\begin{frame}[t]{\ssecIIIa}

\begin{itemize}
%	\item The player who has the lower value may win a first-price sealed-bid auction in Nash eqm.,
%	\item but that this cannot happen in an ascending auction
%	\item The argument depends on asymmetries between bidders.
	\item ``A sledgehammer to crack a nut?''(牛刀割鶏?) 
	\item Vickrey (1961) \\
		--- asymmetries between bidders
	\item Klemperer(1998), $\ldots$ \\
		--- \textsl{almost-common-value} auction
\end{itemize}	
	
\end{frame}

%%%%%%%%%%%%%%%%%%%%%%%%%%%%%%%%%%%%%%%%%%%%%
\begin{frame}[t]{\ssecIIIa}
	
\begin{itemize}
	\item Another reason why first-price sealed-bid auction may be more attractive to entrants:
	\item Sealed-bid auction induces \emph{strategic uncertainty}.
	\vspace{3mm}
	\item Bidders are not likely to share \emph{common prios} about distributions of valuations.
	\begin{itemize}
		\item {\footnotesize With common prior, it cannot be common knowledge that different
		individuals have different beliefs.}
	\end{itemize}
	\item Even if they share common priors, they may not play Nash eqm. strategies.
\end{itemize}	
	
\end{frame}

%%%%%%%%%%%%%%%%%%%%%%%%%%%%%%%%%%%%%%%%%%%%%
\begin{frame}[t]{\ssecIIIa}
	
\begin{itemize}
	\item Over-sensitivity to the significance of information revelation and affiliation \\
		{\footnotesize at the expense of insensitivity to the more important issue of entry.}
	\item \emph{Revelation principle}: any mechanism is equivalent to another mechanism in which agents report their types and wish to do so truthfully.
	\item \emph{Affiliation}: bidders' signals are \textsl{affiliated} if a high value of one bidder's signal makes high values of other bidders' signals more likely.
	\item Netherlands (fiasco), Denmark, UK.
\end{itemize}	
	
\end{frame}

%%%%%%%%%%%%%%%%%%%%%%%%%%%%%%%%%%%%%%%%%%%%%
\newcommand{\ssecIIIb}{4.3.2 Collusion}
\subsection{\ssecIIIb}
\begin{frame}[t]{\ssecIIIb}
	
\begin{itemize}
%	\item Players may behave collusively rather than non-cooperatively.
	\item Collusion in \emph{a multi-unit ascending auction} seems much easier to sustain than in an ordinary industrial market.
	\begin{itemize}
		\item[(1)] Firms can easily identify how to share the collusive division among them.
		\item[(2)] Bids can be used to signal proposals and to signal agreement.
		\item[(3)] Firms' bidding is immediately and perfectly observable, so defection is immediately detected.
		\item[(4)] Punishment is quick, easy and often costless.
		\item[(5)] Entry in an ascending auction may be hard.
	\end{itemize}
\end{itemize}

	
\end{frame}

%%%%%%%%%%%%%%%%%%%%%%%%%%%%%%%%%%%%%%%%%%%%%
\begin{frame}[t]{\ssecIIIb}
	
	\begin{itemize}
		\item \emph{A first-price sealed-bid auction} is usually much more robust to colusion.
		\item Bidders cannot exchange views through their bids.
		\item Bidders cannot observe opponents' bids until after the auction.
		\item Bidders cannot punish defection from any agreement during the course of the auction.
		\item Bidders cannot deter entry easily.
	\end{itemize}
	
	
	
\end{frame}

%%%%%%%%%%%%%%%%%%%%%%%%%%%%%%%%%%%%%%%%%%%%%
\newcommand{\secVI}{4.4 Robustness to Poritical Pressures}
\section{\secVI}
\newcommand{\ssecVIa}{4.4.1\! Economic\!\! Similarity \!$\neq$\! Political\!\! Similarity}
\subsection{\ssecVIa}
\begin{frame}[t]{\ssecVIa}
	
\begin{itemize}
	\item Charging winners different amounts for identical properties might both
	\begin{itemize}
		\item[-] be awkward, and
		\item[-] lead to cautions bidding by managements who did not want to risk the embarrassment of paying more than their rivals.
	\end{itemize}
	\item The gov. would be criticized if the public observed the bids which were very different from the price.  
\end{itemize}
	
\end{frame}

%%%%%%%%%%%%%%%%%%%%%%%%%%%%%%%%%%%%%%%%%%%%%
\begin{frame}[t]{\ssecVIa}
	
\begin{itemize}
	\item The lobbyists' arguments that their suggested change was ``small'' and made the auction more ``standard'', and also that it was ``unfair'', were \emph{politicallly} salient points,
	\item even though they are irrelevant or meaningless from \emph{a stlictly game-theoretic} viewpoint. 
	\item Auction designs that seem similar to economic theorists may seem very different to politicians, bureaucrats and the public,and vice versa.
	\item Political and loibbying presshures need to be predicted and planned for in advance.
\end{itemize}

\section{4.5 (omitted)}
\section{4.6 (omitted)}
\section{4.7 (omitted)}

\end{frame}
	
%%%%%%%%%%%%%%%%%%%%%%%%%%%%%%%%%%%%%%%%%%%%%%
%\newcommand{\secV}{4.5 Understanding the Wider Context}
%\section{\secV}
%\begin{frame}[t]{\secV}
%	
%\begin{itemize}
%	\item 
%\end{itemize}
%	
%\end{frame}



\end{document}